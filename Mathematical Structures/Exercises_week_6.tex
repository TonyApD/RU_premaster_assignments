% caee325d-e3f3-4171-81c1-0e681cdea6bc
% Although we try to provide a template that completely
% matches the corresponding assignment, we do expect you
% to check that you have indeed answered all questions.
%

% ALSO VERY IMPORTANT:
% This is just a template to help you with the LaTeX part of the assignment.
% So you may change it completely according to your own wishes!
%

\documentclass[a4paper]{article}
% Typically the 'article' class is appropriate for assignments.
% And we print it on a4, so we include that as well.

\usepackage{a4wide}
% To decrease the margins and allow more text on a page.

\usepackage{graphicx}
% To deal with including pictures.

\usepackage{enumerate}
% To provide a little bit more functionality than with LaTeX's default
% enumerate environment.

\usepackage{array}
% To provide a little bit more functionality than with LaTeX's default
% array environment.

\usepackage[american]{babel}
% Use this if you want to write the document in US English. It takes care of
% (usually) proper hyphenation.
% If you want to write your answers in Dutch, please replace 'american'
% by 'dutch'.
% Note that after a change it may be that the first compilation of LaTeX
% fails. That is normal and caused by the fact that in auxiliary files
% from previous runs, there may still be a \selectlanguage{american}
% around, which is invalid if 'american' is not incorporated with babel.

\usepackage{amssymb}
% This package loads mathematical things like the fonts for the blackboard
% bold for the set of natural numbers.
\usepackage{amsmath}
% And some student asked me to include amsmath as well...

\usepackage{tikz}
\usetikzlibrary{arrows}
% The tikz package can be used to draw all kinds of diagrams.
% In this assignment it is being used for drawing the parse trees.

\usepackage[all]{xy}
% Instead of tikz you can also use xy to draw parse trees with


\usepackage{xspace}
% xspace can be used to let LaTeX decide whether a command should be followed
% y a space or not, depending on what follows.

% Some obscure definition to create a circled node within xy.
% The definition is made by Freek Wiedijk who prefers to do his definitions
% in TeX instead of LaTeX, which explains the \def instead of \newcommand.
\def\node{*++[o][F-]}
\def\fnode{*++[o][F=]}

\newcommand{\exercise}[2]{\subsection*{Exercise #1}{#2}}
\newcommand{\exerciseenum}[2]{\subsection*{Exercise #1}{\begin{enumerate}[a)]#2\end{enumerate}}}
% We defined our own commands to make it easy to present all the
% exercises in the same style. The first one does not automatically
% start an 'enumerate' list, the second one does.
% The [2] means that our command needs two arguments.
% The #1 and the #2 indicate where we use these arguments in the
% command.
% There are several ways to have automatic numbering for the exercises,
% but here we have chosen to use a subsection for this and use manual
% numbering. This is because maybe not everyone will be able to do hand in
% all exercises.
% Note that we add the '*' to make sure that the subsection is not numbered.
% (Since we don't have a \section, the numbers for a subsection would be
% ugly like 0.1, 0.2 et cetera.
% The environment 'enumerate' automatically numbers the items in this list.
% The optional [a)] makes sure that the list will be like a), b), c) et cetera.

\usepackage{mathtools}

\newcommand{\abs}[1]{\ensuremath{\left|\, #1 \,\right|}}
\newcommand{\floor}[1]{\ensuremath{\left\lfloor\, #1 \,\right\rfloor}}
\newcommand{\ceil}[1]{\ensuremath{\left\lceil\, #1 \,\right\rceil}}
% Abbreviations for the absolute value, ceil and floor function.

\newcommand{\set}[1]{\ensuremath{\left\{{#1}\right\}}}
% This command puts curly braces around its argument, so it becomes
% a set. The \left and \right make sure that the braces grow in size
% if the contents of the set are large symbols.

\newcommand{\setbuild}[2]{\ensuremath{\set{{#1}\mid{#2}}}}
% We also introduce a shortcut for using the set builder notation.
% Do you understand what it does?

\newcommand{\seq}[1]{\ensuremath{\left\{{#1}\right\}}}
% This puts curly braces to define a sequence.
% Note that this is the same as the definition for a \set.

% And the next series of commands gives you some of the default sets
% that were in the slides.
\newcommand{\TT}{\ensuremath{\mathbb{T}}}
\newcommand{\FF}{\ensuremath{\mathbb{F}}}
\newcommand{\NN}{\ensuremath{\mathbb{N}}}
\newcommand{\NNp}{\ensuremath{\mathbb{N}^{+}}}
\newcommand{\ZZ}{\ensuremath{\mathbb{Z}}}
\newcommand{\ZZp}{\ensuremath{\mathbb{Z}^{+}}}
\newcommand{\QQ}{\ensuremath{\mathbb{Q}}}
\newcommand{\QQp}{\ensuremath{\mathbb{Q}^{+}}}
\newcommand{\RR}{\ensuremath{\mathbb{R}}}
\newcommand{\RRp}{\ensuremath{\mathbb{R}^{+}}}
\newcommand{\CC}{\ensuremath{\mathbb{C}}}

% And the next command gives a shorthand for the power set of a given set.
\newcommand{\power}[1]{\ensuremath{{\cal P}\left({#1}\right)}}

% The following two commands can be used to get an upright T or F, even
% when in math mode.
\newcommand{\Tt}{\ensuremath{\mathrm T}}
\newcommand{\Ff}{\ensuremath{\mathrm F}}

% Below is the tikz-definition that is used for the parse trees.
\tikzset{
  treenode/.style = {
    align=center,
    inner sep=0pt,
    text centered,
    font=\sffamily},
  arn_n/.style = {
    thick,
    treenode,
    circle,
    font=\sffamily\bfseries,
    draw=black,
    %text width=1.5em,
    minimum size=1.5em
    },
  arn_r/.style = {
    treenode,
    circle,
    red,
    draw=red,
    %text width=1.5em,
    minimum size=1.5em,
    very thick},
}

% We want to have a good way of presenting the | relation.
\DeclareMathOperator{\divides}{\mid}

% And we also want to have an easy way to create a bold mod as operator.
\DeclareMathOperator{\emod}{\mathbf{mod}}

% And in addition we want to have a non bold version for (mod n).
\newcommand{\tmod}{\mbox{mod}\xspace}

% We create an environment for numbered theorems.
\newtheorem{theorem}{Theorem}

% And we create a \templtag command for referring to the template.
\newcommand{\templtag}[1]{\marginpar{\fbox{#1}}}
\reversemarginpar
\title{Mathematical Structures\\Assignment 6}

% Replace the placeholders by your real name, student number and
% group (for the exercise hours)
\author{Tony Lopar \\ s1013792 \quad Group 1}

% In LaTeX everything before \begin{document} is called pre-amble.
% This is where you put all important settings. The real document
% starts after \begin{document}.
\begin{document}
\maketitle
% \maketitle makes sure that the title is shown on the first page of
% the document.


% Now we use the command we defined earlier and give it the proper two
% parameters.
% Because the second parameter is long, we put a % directly after the
% opening curly brace {. This is not needed but makes the source file
% look a bit better.
\exercise{10}{%
% Remove the part that you don't need!
% Either
% \begin{quote}
% We are going to prove the statement.
% So we may assume that $a \divides bc$, where $a$, $b$, and $c$
% are positive integers and $a \neq 0$,
% and we show that from this it follows that $a \divides b$ or $a \divides c$.
% \dots
% \end{quote}
% or
\begin{quote}
We are going to disprove the statement.
So we are going to give a counterexample such that a = 6, b = 3 and c = 4. In this example we will get that $6 \mid 3 \cdot 4$, but $6 \nmid 3$ and $6 \nmid 4$, so the statement is false.
\end{quote}
}
\exerciseenum{11}{%
\item%a
\begin{eqnarray*}
\lefteqn{(99^2 \emod 32)^3 \emod 15 } \\ % You may use \lefteqn{\dots}
                                         % if you want the left hand side
                                         % to overflow above the middle
                                         % and right hand side of the array.
&=& (9801 \emod 32)^3 \emod 15  \\
&=& 9^3 \emod 15 \\
&=& 729 \emod 15 \\
&=& 9
\end{eqnarray*}
}

\exercise{12}{%
We may assume that $n$ is an integer
and we will show that from this it follows that
$n^2 \equiv 0 \, (\tmod\, 4)$ % The \, provides a bit extra space.
or
$n^2 \equiv 1 \, (\tmod\, 4)$.
We start by making a case distinction:

\begin{equation}
    n^2 \equiv
    \begin{cases*}
        0(\emod 4)
        % , & \text{for } n \enspace is \enspace even   \\
        \enspace \text{or} \\
        1(\emod 4)
        % , & \text{for } n \enspace is \enspace odd
    \end{cases*}
  \end{equation}
  From this case distinction we have to find in which cases $n^2$ is congruent to that case. We find that when n is an even number, then it it can also be written as $n = 2k$ where $k \in  \ZZ$. So $n^2 = 4k^2$, which is dividable by 4, so is congruent to $0 (\emod 4)$. Since an even number is 2k, we know that 2k + 1 or 2k - 1 are both odd numbers. Let's first take a look at $n = 2k + 1$ which will be $n^2 = 4k^2 + 4k + 1$. In this case $(4k^2 + 4k + 1)(\emod 4) = 1$. In the case of $n = 2k - 1$ we find that $n^2 = 4k^2 - 4k + 1 \equiv 1(\emod 4)$. So, we discovered the following case distinction:
\begin{equation}
    n^2 \equiv
    \begin{cases*}
        0(\emod 4), & \text{for n is even}   \\
        1(\emod 4), & \text{for n is odd}
    \end{cases*}
  \end{equation}
}

\exercise{13}{%
We use the notation used in the slides:
\[
\begin{array}{c|c|c|l}
\mbox{step} & \mbox{pair} & \mbox{remainder} & \mbox{linear combination}
\\ \hline
1 & (91,26)     & 13 = 91 - 3 \cdot 26  & 13 = 1 \cdot 91 - 3 \cdot 26 \\
  &             &                       & 13 = 0 \cdot 26 + 1 \cdot (91 - 3 \cdot 26) \\
2 & (26, 13)    & 0 = 26 - 2 \cdot 13       & 13 = 0 \cdot 26 + 1 \cdot 13
\end{array}
\]
}

\exercise{14}{%
I will solve this exercise by using trial division. Since a and b contain integers that aren't primes I will split these in primes. For example, $8^3$ will become $2^3 \cdot (2^2)^3$. This results in the following values: \\
$a = 2^9 \cdot 3^7 \cdot 5^6$ \\
$b = 2^3 \cdot 3^5 \cdot 7^4 \cdot 11$ \\
Now we know this, we can calculate gcd(a, b) which is $2^3 \cdot 3^5 = 1944$. We can also calculate the lcm(a, b) which is $2^9 \cdot 3^7 \cdot 5^6 \cdot 7^4 \cdot 11 = 462086856000000$.
% We could use the Euclidean algorithm again, but there is a smarter way
% to do this.
}

\exercise{15}{%
First of all, we set the domain of all integers in this exercise to $\ZZ_{11}$, so the $\emod11$ after each integer will be left out.
\begin{eqnarray*}
7^{1234} &=& 7^{123 \cdot 10 + 4} \\
&=& 7^{123 \cdot 10} \cdot 7^4 \\
&=& (7^{123})^{10} \cdot 7^4 \\
&\stackrel{\star}{=}& 1 \cdot 7^4 \\
&=& 7^4 \\
&\stackrel{\star}{=}& 7(\emod 4) \\
&=& 3
\end{eqnarray*}
At $(\star)$ we use Fermat's little theorem,
which is allowed because in the first case $11 \in \ZZ$ which is also a prime and $11 \nmid 7$. In the second case we can use Fermat's little theorem, because 7 is an integer.
}

\exercise{16}{%
We can transform this question into the following simultaneous
system of equations:
\begin{eqnarray*}
n &\equiv & 2 \, (\tmod 3)\\
n &\equiv & 4 \, (\tmod 5)\\
n &\equiv & 5 \, (\tmod 7)\\
\end{eqnarray*}
By knowing this we can assign the following variables:
\begin{eqnarray*}
    m_1 &:=& 3 \\
    m_2 &:=& 5 \\
    m_3 &:=& 7 \\
    m &:=& 3 \cdot 5 \cdot 7 = 105 \\
    M_1 &:=& \frac{105}{3} = 35 \\
    M_2 &:=& \frac{105}{5} = 21 \\
    M_3 &:=& \frac{105}{7} = 15 \\
\end{eqnarray*}

Then gcd($m_1, M_1) = 1$, gcd($m_2, M_2) = 1$ and gcd($m_3, M_3) = 1$.
Thus we need to find an $y_1$ such that $M_1 \cdot y_1 \equiv 1 (\emod m)$, an $y_2$ such that $M_2 \cdot y_2 \equiv 1 (\emod m)$ and an $y_3$ such that $M_3 \cdot y_3 \equiv 1 (\emod m)$. \\
In order to find $y_1$ we have to discover with which value $35 \cdot y_1 \equiv 1 (\emod 3)$. By trying some values for $y_1$, we discover that 2 is a suitable value, because $70 (\emod 3) = 1$. \\
To find $y_2$ we have to discover with which value $21 \cdot y_2 \equiv 1 (\emod 5)$. By trying some values for $y_2$, we discover that 1 is a suitable value, because $21 (\emod 5) = 1$. \\
To find $y_3$ we have to discover with which value $15 \cdot y_3 \equiv 1 (\emod 7)$. By trying some values for $y_3$, we discover that 8 is a suitable value, because $120 (\emod 7) = 1$. \\

Now we know the values for y, we can calculate our value for n.
\begin{eqnarray*}
    n &:=& (2 \cdot M_1 \cdot y_1 + 4 \cdot M_2 \cdot y_2 + 5 \cdot M_3 \cdot y_3) \emod m \\
        &=& (2 \cdot 35 \cdot 2 + 4 \cdot 21 \cdot 1 + 5 \cdot 15 \cdot 8)\emod 105 \\
        &=& (140 + 84 + 1000)\emod 105 \\
        &=& (1224)\emod 105 \\
        &=& 79 \\
\end{eqnarray*}
All solutions in $\ZZ$ are given by the set $\{79 + k \cdot 105 \enspace | \enspace k \in \ZZ \}$.

}
\end{document}
