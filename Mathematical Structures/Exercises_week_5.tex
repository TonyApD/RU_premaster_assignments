% The submission confirmation number is d6c6dab5-0ab7-46a1-baba-46017dee087c
% Although we try to provide a template that completely
% matches the corresponding assignment, we do expect you
% to check that you have indeed answered all questions.
%

% ALSO VERY IMPORTANT:
% This is just a template to help you with the LaTeX part of the assignment.
% So you may change it completely according to your own wishes!
%

\documentclass[a4paper]{article}
% Typically the 'article' class is appropriate for assignments.
% And we print it on a4, so we include that as well.

\usepackage{a4wide}
% To decrease the margins and allow more text on a page.

\usepackage{graphicx}
% To deal with including pictures.

\usepackage{enumerate}
% To provide a little bit more functionality than with LaTeX's default
% enumerate environment.

\usepackage{array}
% To provide a little bit more functionality than with LaTeX's default
% array environment.

\usepackage[american]{babel}
% Use this if you want to write the document in US English. It takes care of
% (usually) proper hyphenation.
% If you want to write your answers in Dutch, please replace 'american'
% by 'dutch'.
% Note that after a change it may be that the first compilation of LaTeX
% fails. That is normal and caused by the fact that in auxiliary files
% from previous runs, there may still be a \selectlanguage{american}
% around, which is invalid if 'american' is not incorporated with babel.

\usepackage{amssymb}
% This package loads mathematical things like the fonts for the blackboard
% bold for the set of natural numbers.
\usepackage{amsmath}
% And some student asked me to include amsmath as well...

\usepackage{tikz}
\usetikzlibrary{arrows}
% The tikz package can be used to draw all kinds of diagrams.
% In this assignment it is being used for drawing the parse trees.

\usepackage[all]{xy}
% Instead of tikz you can also use xy to draw parse trees with

% Some obscure definition to create a circled node within xy.
% The definition is made by Freek Wiedijk who prefers to do his definitions
% in TeX instead of LaTeX, which explains the \def instead of \newcommand.
\def\node{*++[o][F-]}
\def\fnode{*++[o][F=]}

\newcommand{\exercise}[2]{\subsection*{Exercise #1}{#2}}
\newcommand{\exerciseenum}[2]{\subsection*{Exercise #1}{\begin{enumerate}[a)]#2\end{enumerate}}}
% We defined our own commands to make it easy to present all the
% exercises in the same style. The first one does not automatically
% start an 'enumerate' list, the second one does.
% The [2] means that our command needs two arguments.
% The #1 and the #2 indicate where we use these arguments in the
% command.
% There are several ways to have automatic numbering for the exercises,
% but here we have chosen to use a subsection for this and use manual
% numbering. This is because maybe not everyone will be able to do hand in
% all exercises.
% Note that we add the '*' to make sure that the subsection is not numbered.
% (Since we don't have a \section, the numbers for a subsection would be
% ugly like 0.1, 0.2 et cetera.
% The environment 'enumerate' automatically numbers the items in this list.
% The optional [a)] makes sure that the list will be like a), b), c) et cetera.



\newcommand{\abs}[1]{\ensuremath{\left|\, #1 \,\right|}}
\newcommand{\floor}[1]{\ensuremath{\left\lfloor\, #1 \,\right\rfloor}}
\newcommand{\ceil}[1]{\ensuremath{\left\lceil\, #1 \,\right\rceil}}
% Abbreviations for the absolute value, ceil and floor function.

\newcommand{\set}[1]{\ensuremath{\left\{{#1}\right\}}}
% This command puts curly braces around its argument, so it becomes
% a set. The \left and \right make sure that the braces grow in size
% if the contents of the set are large symbols.

\newcommand{\setbuild}[2]{\ensuremath{\set{{#1}\mid{#2}}}}
% We also introduce a shortcut for using the set builder notation.
% Do you understand what it does?

\newcommand{\seq}[1]{\ensuremath{\left\{{#1}\right\}}}
% This puts curly braces to define a sequence.
% Note that this is the same as the definition for a \set.

% And the next series of commands gives you some of the default sets
% that were in the slides.
\newcommand{\TT}{\ensuremath{\mathbb{T}}}
\newcommand{\FF}{\ensuremath{\mathbb{F}}}
\newcommand{\NN}{\ensuremath{\mathbb{N}}}
\newcommand{\NNp}{\ensuremath{\mathbb{N}^{+}}}
\newcommand{\ZZ}{\ensuremath{\mathbb{Z}}}
\newcommand{\ZZp}{\ensuremath{\mathbb{Z}^{+}}}
\newcommand{\QQ}{\ensuremath{\mathbb{Q}}}
\newcommand{\QQp}{\ensuremath{\mathbb{Q}^{+}}}
\newcommand{\RR}{\ensuremath{\mathbb{R}}}
\newcommand{\RRp}{\ensuremath{\mathbb{R}^{+}}}
\newcommand{\CC}{\ensuremath{\mathbb{C}}}

% And the next command gives a shorthand for the power set of a given set.
\newcommand{\power}[1]{\ensuremath{{\cal P}\left({#1}\right)}}

% The following two commands can be used to get an upright T or F, even
% when in math mode.
\newcommand{\Tt}{\ensuremath{\mathrm T}}
\newcommand{\Ff}{\ensuremath{\mathrm F}}

% Below is the tikz-definition that is used for the parse trees.
\tikzset{
  treenode/.style = {
    align=center,
    inner sep=0pt,
    text centered,
    font=\sffamily},
  arn_n/.style = {
    thick,
    treenode,
    circle,
    font=\sffamily\bfseries,
    draw=black,
    %text width=1.5em,
    minimum size=1.5em
    },
  arn_r/.style = {
    treenode,
    circle,
    red,
    draw=red,
    %text width=1.5em,
    minimum size=1.5em,
    very thick},
}

% We create an environment for numbered theorems.
\newtheorem{theorem}{Theorem}

% And we create a \templtag command for referring to the template.
\newcommand{\templtag}[1]{\marginpar{\fbox{#1}}}
\reversemarginpar
\title{Mathematical Structures\\Assignment 5}

% Replace the placeholders by your real name, student number and
% group (for the exercise hours)
\author{Tony Lopar \\ s1013792 \quad Group 1}

% In LaTeX everything before \begin{document} is called pre-amble.
% This is where you put all important settings. The real document
% starts after \begin{document}.
\begin{document}
\maketitle
% \maketitle makes sure that the title is shown on the first page of
% the document.


% Now we use the command we defined earlier and give it the proper two
% parameters.
% Because the second parameter is long, we put a % directly after the
% opening curly brace {. This is not needed but makes the source file
% look a bit better.
\exerciseenum{9}{%
\item%a
We start by building a table with the first values of the sequence:
\[
\renewcommand*{\arraystretch}{1.3}
\begin{array}{|c|c|c|}
\hline
n & {\sum_{i=1}^n \frac{1}{i(i+1)}} & = \\
\hline
1 & \frac{1}{1\cdot 2}  & \frac{1}{2}      \\
2 & \frac{1}{1\cdot 2}+\frac{1}{2\cdot 3} & \frac{2}{3}       \\
3 & \frac{1}{1\cdot 2}+\frac{1}{2\cdot 3}+\frac{1}{3\cdot 4} & \frac{3}{4} \\
n & \frac{1}{1\cdot 2}+\frac{1}{2\cdot 3}+\frac{1}{3\cdot 4}+\dots+\frac{1}{n \cdot (n+1)} & \frac{n}{n+1} \\
\hline
\end{array}
\]

\item%b
We give a proof by mathematical induction following the template:
\setcounter{theorem}{8}% One less, because \begin{theorem} starts by adding 1.
                       % And we want the number to coincide with the exercise
                       % number.
% Note that you don't have to label your proof with these \templtag-s,
% but if you keep them in, make sure that you have them at the right place.
\begin{theorem}
Let \templtag{1}%
the predicate $P:\ZZp\to\set{\Tt,\Ff}$ be defined by:
\[
P(n) := \sum_{i=1}^n \frac{1}{i(i+1)}= \frac{n}{n+1}
\]
Then $P(n)$ holds for all $n\in\ZZp$ with $n\geq 1$.
\end{theorem}

Proof \templtag{2}%
by induction on $n$.
\begin{description}
\item[Basis step]~\templtag{3}\\

\medskip % These skips are here to make the tags stand out in the margin.
$P(1)$ holds, \templtag{4}%
because $\frac{1}{1 \cdot (1 + 1)} = \frac{1}{1 \cdot 2} = \frac{1}{2} = \frac{1}{1 + 1}$
\item[Inductive step]~\templtag{5}\\

\medskip
\begin{itemize}
\item
Let \templtag{6}%
$k\in\ZZp$ with $k\geq 1$ such that $P(k)$ holds.

\medskip
\item
Hence \templtag{7}%
$\frac{1}{1\cdot 2}+\frac{1}{2\cdot 3}+\frac{1}{3\cdot 4}+\dots+\frac{1}{k \cdot (k+1)} = \frac{k}{k + 1}$ (\mbox{IH}).

\medskip
\item
We \templtag{8}%
have to prove that $P(k + 1)$ holds.

\medskip
\item
Hence \templtag{9}%
we have to prove
$\frac{1}{1\cdot 2}+\frac{1}{2\cdot 3}+\frac{1}{3\cdot 4}+\dots+\frac{1}{k \cdot (k+1)} + \frac{1}{(k+1) \cdot ((k + 1)+1)} = \frac{k + 1}{(k + 1) + 1}$
$\sum_{i=1}^{k + 1} \frac{1}{i(i+1)}  = \frac{k + 1}{(k + 1) + 1}$

\medskip
\item
This \templtag{10}%
holds because:
\begin{eqnarray*}
\sum_{i=1}^{k + 1} \frac{1}{i(i+1)}
&=& \sum_{i=1}^{k} \frac{1}{i(i+1)} + \frac{1}{(k + 1)((k + 1) + 1)}  \\
&\stackrel{\small\mbox{IH}}{=} & \frac{k}{k + 1} + \frac{1}{(k+1)((k + 1) + 1)} \\
&=    & \frac{k}{k + 1} + \frac{1}{(k+1)(k + 2)} \\
&=    & \frac{k(k + 2)}{(k + 1)(k + 2)} + \frac{1}{(k+1)(k + 2)} \\
&=    & \frac{k(k + 2) + 1}{(k + 1)(k + 2)} \\
&=    & \frac{k^2 + 2k + 1}{(k + 1)(k + 2)} \\
&=    & \frac{(k + 1)^2}{(k + 1)(k + 2)} \\
&=    & \frac{k + 1}{k + 2} \\
&=    & \frac{k + 1}{(k + 1) + 1}
\end{eqnarray*}

\medskip
This \templtag{11}%
completes the Inductive step.
\end{itemize}
\end{description}

\medskip
The \templtag{12}%
Principle of mathematical induction now tells us that P(n) holds for all $n \in \mathbb{Z}^+$ where $n \geq 1$
}

\exercise{10}{%
\setcounter{theorem}{9}% One less, because \begin{theorem} starts by adding 1.
                       % And we want the number to coincide with the exercise
                       % number.
% Note that you don't have to label your proof with these \templtag-s,
% but if you keep them in, make sure that you have them at the right place.
\begin{theorem}
Let \templtag{1}%
the predicate $P:\setbuild{n\in \ZZp}{n \mbox{~is an odd integer} }\to\set{\Tt,\Ff}$ be defined by:
\[
P(n) := n^2 - 1 = 8m \enspace where \enspace m \in \NN
\]
Then $P(n)$ holds for all $n\in\setbuild{n\in \ZZp}{k \mbox{~is an odd integer} }$ with $n\geq 1$.
\end{theorem}

Proof \templtag{2}%
by induction on $n$.
\begin{description}
\item[Basis step]~\templtag{3}\\

\medskip % These skips are here to make the tags stand out in the margin.
$P(1)$ holds, \templtag{4}%
because $\frac{1}{1 \cdot (1 + 1)} = \frac{1}{1 \cdot 2} = \frac{1}{2} = \frac{1}{1 + 1}$
\item[Inductive step]~\templtag{5}\\

\medskip
\begin{itemize}
\item
Let \templtag{6}%
$k\in\setbuild{k\in \ZZp}{k \mbox{~is an odd integer} }$ with $k\geq 1$ such that $P(k)$ holds.

\medskip
\item
Hence \templtag{7}%
$\sum^{k}_{i = 1}(k^2 - 1 = 8m$ for some integer $m \in \NN$) (\mbox{IH}).

\medskip
\item
We \templtag{8}%
have to prove that $P(k + 2)$ holds, since we have to take the next odd integer from the odd integer $k \in \ZZp$.

\medskip
\item
Hence \templtag{9}%
we have to prove
$k^2 - 1 + (k + 2)^2 - 1 = 8m$

\medskip
\item
This \templtag{10}%
holds because:
\begin{eqnarray*}
k^2 - 1 = 8m  \\
k^2 = 8m + 1  \\
8m \\
&\stackrel{\small\mbox{IH}}{=} & k^2 - 1 + (k + 2)^2 - 1  \\
&=    & k^2 - 1 + k^2 + 2k + 4 - 1  \\
&=    & 2k^2 + 2k + 2 \\
&=    & 2(8m + 1) + 2k + 2 \\
&=    & 16m + 2 + 2k + 2 \\
&=    & 16m + 2k + 4 \\
&=    & 8(2m + \frac{1}{4}k + \frac{1}{2}) \\
\end{eqnarray*}

\medskip
This \templtag{11}%
completes the Inductive step.
\end{itemize}
\end{description}

\medskip
The \templtag{12}%
Principle of mathematical induction now tells us that P(n) holds for all $n \in \setbuild{n\in \ZZp}{n \mbox{~is an odd integer} }$ where $n \geq 1$
}

\exercise{11}{%
We give a proof by strong induction.
Therefore we have adjusted the default template to match the principle of
strong induction as explained on page 332 of the book.
% Page 332 in the real book, not in the PDF.
% On this page there are some important variables b and j mentioned.
% What are their values in this specific case?

\setcounter{theorem}{10}% One less, because \begin{theorem} starts by adding 1.
                       % And we want the number to coincide with the exercise
                       % number.
\begin{theorem}
Let \templtag{1}%
the predicate $P: \NN \to \{T, F \}$ be defined by: \\
\[5^n = \alpha^2 + \beta^2 \enspace where \enspace \alpha, \beta \in \NN \]
Then $P(n)$ holds for all $n\in\NN$ with $n\geq 0$.
\end{theorem}

Proof \templtag{2}%
by induction on $n$.
\begin{description}
\item[Basis step]~\templtag{3}\\

\medskip % These skips are here to make the tags stand out in the margin.
$P(0)$ holds, \templtag{4}%
because $5^0 = 1 = 0^2 + 1^2$

\item[Inductive step]~\templtag{5}\\

\medskip
\begin{itemize}
\item
Let \templtag{6}%
$k\in\NN$ with $k\geq 0$ such that $P(k)$ holds.

\medskip
\item
Hence \templtag{7}%
$5^k = \alpha^2 + \beta^2$ (\mbox{IH}).

\medskip
\item
We \templtag{8}%
have to prove that $P(k + 1)$ holds

\medskip
\item
Hence \templtag{9}%
we have to prove
$5^{k+1} = \alpha^2 + \beta^2$ where $\alpha, \beta \in \NN$ \\

\medskip
\item
This \templtag{10}%
holds because:

\[
5^{k+1} = \alpha^2 + \beta^2 \\
\] \[
5^k \cdot5^1 = \alpha^2 + \beta^2 \\
\] \[
5^k = \frac{\alpha^2 + \beta^2}{5} \\
\] \[
5^k = \frac{5\alpha^2 \cdot 5\beta^2}{5^2} \\
\]



\medskip
This \templtag{11}%
completes the Inductive step, since we know that $5^k$ is true.
\end{itemize}
\end{description}

\medskip
}


\end{document}
