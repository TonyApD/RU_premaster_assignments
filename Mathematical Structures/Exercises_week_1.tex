%
% Although we try to provide a template that completely
% matches the corresponding assignment, we do expect you
% to check that you have indeed answered all questions.
%

% ALSO VERY IMPORTANT:
% This is just a template to help you with the LaTeX part of the assignment.
% So you may change it completely according to your own wishes!
%

\documentclass[a4paper]{article}
% Typically the 'article' class is appropriate for assignments.
% And we print it on a4, so we include that as well.

\usepackage{a4wide}
% To decrease the margins and allow more text on a page.

\usepackage{graphicx}
% To deal with including pictures.

\usepackage{enumerate}
% To provide a little bit more functionality than with LaTeX's default
% enumerate environment.

\usepackage{array}
% To provide a little bit more functionality than with LaTeX's default
% array environment.

\usepackage[american]{babel}
% Use this if you want to write the document in US English. It takes care of
% (usually) proper hyphenation.
% If you want to write your answers in Dutch, please replace 'american'
% by 'dutch'.
% Note that after a change it may be that the first compilation of LaTeX
% fails. That is normal and caused by the fact that in auxiliary files
% from previous runs, there may still be a \selectlanguage{american}
% around, which is invalid if 'american' is not incorporated with babel.

\usepackage{amssymb}
% This package loads mathematical things like the fonts for the blackboard
% bold for the set of natural numbers.

\newcommand{\exercise}[2]{\subsection*{Exercise #1}{#2}}
\newcommand{\exerciseenum}[2]{\subsection*{Exercise #1}{\begin{enumerate}[a)]#2\end{enumerate}}}
% We defined our own commands to make it easy to present all the
% exercises in the same style. The first one does not automatically
% start an 'enumerate' list, the second one does.
% The [2] means that our command needs two arguments.
% The #1 and the #2 indicate where we use these arguments in the
% command.
% There are several ways to have automatic numbering for the exercises,
% but here we have chosen to use a subsection for this and use manual
% numbering. This is because maybe not everyone will be able to do hand in
% all exercises.
% Note that we add the '*' to make sure that the subsection is not numbered.
% (Since we don't have a \section, the numbers for a subsection would be
% ugly like 0.1, 0.2 et cetera.
% The environment 'enumerate' automatically numbers the items in this list.
% The optional [a)] makes sure that the list will be like a), b), c) et cetera.


\newcommand{\set}[1]{\ensuremath{\left\{{#1}\right\}}}
% This command puts curly braces around its argument, so it becomes
% a set. The \left and \right make sure that the braces grow in size
% if the contents of the set are large symbols.

\newcommand{\setbuild}[2]{\ensuremath{\set{{#1}\mid{#2}}}}
% We also introduce a shortcut for using the set builder notation.
% Do you understand what it does?

% And the next series of commands gives you some of the default sets
% that were in the slides.
\newcommand{\TT}{\ensuremath{\mathbb{T}}}
\newcommand{\FF}{\ensuremath{\mathbb{F}}}
\newcommand{\NN}{\ensuremath{\mathbb{N}}}
\newcommand{\NNp}{\ensuremath{\mathbb{N}^{+}}}
\newcommand{\ZZ}{\ensuremath{\mathbb{Z}}}
\newcommand{\ZZp}{\ensuremath{\mathbb{Z}^{+}}}
\newcommand{\QQ}{\ensuremath{\mathbb{Q}}}
\newcommand{\QQp}{\ensuremath{\mathbb{Q}^{+}}}
\newcommand{\RR}{\ensuremath{\mathbb{R}}}
\newcommand{\RRp}{\ensuremath{\mathbb{R}^{+}}}
\newcommand{\CC}{\ensuremath{\mathbb{C}}}

% An the next command gives a shorthand for the power set of a given set.
\newcommand{\power}[1]{\ensuremath{{\cal P}\left({#1}\right)}}

\title{Mathematical Structures\\Assignment 1}

% Replace the placeholders by your real name, student number and
% group (for the exercise hours)
\author{Tony Lopar \\ s1013792 \quad Group 1}

% In LaTeX everything before \begin{document} is called pre-amble.
% This is where you put all important settings. The real document
% starts after \begin{document}.
\begin{document}
\maketitle
% \maketitle makes sure that the title is shown on the first page of
% the document.


% Now we use the command we defined earlier and give it the proper two
% parameters.
% Because the second parameter is long, we put a % directly after the
% opening curly brace {. This is not needed but makes the source file
% look a bit better.
\exerciseenum{16}{%
\item%a
The set $\{0,3,6,9,12\}$
% If we want math then we have to use $...$ or \[...\] to change from text
% mode to math mode.
can be written using set builder notation as follows:
\[
\setbuild{q\in \NN}{q \mbox{~is a multiple of 3 and $q < 13$} } \\
\]
The set starts with 0 and is increased every step by 3, so every element is a multiplication of 3.
% If you are in math mode, but you want to use normal text, you can do that
% using the \mbox{...} command. The text within the curly braces is typeset
% as normal text. Note the ~ in the beginning. This is because LaTeX doesn't
% add spaces in math mode between the q and the text.

% Obviously the answer above is wrong, but now you have an example
% of how to do this.
% If you don't know the answer, then please remove this stupid answer
% because then the assistants don't have to wonder why you gave such a
% stupid answer.

% Because we don't want to hand in b) we tell enumerate (enum) to skip 1
% number on the first level (i).
\addtocounter{enumi}{1}
\item%c
The set $\{m,n,o,p\}$
can be written using set builder notation as follows:
\[
\setbuild{c}{c \mbox{~is an lowercase letter in the Latin alphabet after l and before q}}
\]
The letters in the set are all lowercase and are in the Latin alphabet. The shown letters are between l and q.

% You will see that if you uncomment the next \item, it will show as d).
%\item%d
}

\exerciseenum{17}{%
\item%a
The sets $\set{1,3,3,3,5,5,5,5,5,}$ and $\set{5,3,1}$ are equal,
because they contain exactly the same elements, namely 1, 3 and 5. The order and number of occurrences of elements doesn't matter when comparing sets, which makes the sets equal.
\item%b
The sets $\set{\set{1}}$ and $\set{1,\set{1}}$ are not equal,
because $1 \notin \set{\set{1}}$ while $1 \in \set{1,\set{1}}$
}

\exerciseenum{18}{%
\item%a
The statement
$0 \in \emptyset$
is false,
because an $\emptyset$ has no elements at all, so 0 can't be an element from the set.

\addtocounter{enumi}{2}
\item%d
The statement
$\emptyset\subset \set{0}$
is true,
because the $\emptyset$ is a proper subset of every nonempty set.

\addtocounter{enumi}{2}
\item%g
The statement
$\set{x}\in\set{x}$
is false,
because $\set{x}$ doesn't contain a set of x as element.
}

\exercise{19}{%
If $A$ and $B$ are arbitrary sets such that $\power{A}=\power{B}$,
then it
follows
that $A=B$,
because if we assume that $A \neq B$, then A or B should have an element that isn't an element in the other set. If we have $x \in A$ and $x \notin B$, then $\set{x} \in \power{A}$ while $\set{x} \notin \power{B}$. The element $\set{x}$ is in $\power{A}$, but not in $\power{B}$, so $\power{A} \neq \power{B}$. So if $P(A) = P(B)$, then we need two sets with the same subsets which means $A = B$ needs to be true.
}

\exerciseenum{20}{%
\item%a
There
exists
a set $A$ such that $\power{A}=\emptyset$,
because if we take $A = \emptyset$ the resulting $\power{A}$ will only contain $\emptyset$
\item%b
There
exists
a set $A$ such that $\power{A}=\set{\emptyset, \set{a}}$,
because if we take $A = \set{a}$ the powerset will be: $\power{A} = \set{\emptyset, \set{a}}$.
}

\exercise{21}{%
Let's assume that $A$ and $B$ are both non-empty sets such that $A\neq B$.
Then $A \times B$ will have the cartesian product: $\set{(A_1, B_1), (A_1, B_2), (A_2, B_1), (A_2, B_2)}$ while $B \times A = \set{(B_1, A_1), (B_1, A_2), (B_2, A_1), (B_2, A_2)}$. Since the position of elements matters in cartesian products these sets are not equal if A en B are not equal.
}

\exerciseenum{22}{%
\item%a
$A\cup B = \set{a, b, c, d, e, f, g, h}$

\addtocounter{enumi}{2}
\item%d
$A - B = \set{f, g, h}$
}

\exerciseenum{23}{%
\addtocounter{enumi}{1}
\item%b
If we assume that $A$ and $B$ are arbitrary sets such that $A\subseteq B$,
then $A\cap B = A$ holds.
This is because if we assume that $A = \set{1, 2, 3, 4}$ and $B = \set{1, 2, 3, 4, 5, 6}$, then $A \subset B$ applies. Because $5 \notin A$ and $6 \notin A$ the intersect will be as follows: $A \cap B = \set{1, 2, 3, 4}$. This is equal to set A.
}

\exerciseenum{24}{%
\addtocounter{enumi}{3}
\item%d
Let $A_i=(i,\infty)=\setbuild{x\in\RR}{x>i}$.
Then
\begin{eqnarray*}
A_1 &=& (1,\infty) \\
A_2 &=& (2,\infty) \\
    &\vdots&
\end{eqnarray*}
This leads to the following observations: the first value of the pair increases for every i which decreases the range of the interval with 1 every step. $(i, \infty)$ is an open interval which means i and $\infty$ are not included. Because i starts at 1 will never included in the interval. The union of $A_i$ contains all integers between 1 and $\infty$. Because the range is shrinking as $A_i$ rises the last possible set will be an empty set and because $\emptyset$ is an subset of every set, this will be the intersect.

And from this we may conclude that
\begin{eqnarray*}
\bigcup_{i=1}^{\infty} A_i &=& \mathbb{Z^+} - 1 \\
\bigcap_{i=1}^{\infty} A_i &=& \emptyset
\end{eqnarray*}

}
\end{document}
