%The submission confirmation number is a2de2f37-7748-4ba2-9090-84fa87170f9a
% Although we try to provide a template that completely
% matches the corresponding assignment, we do expect you
% to check that you have indeed answered all questions.
%

% ALSO VERY IMPORTANT:
% This is just a template to help you with the LaTeX part of the assignment.
% So you may change it completely according to your own wishes!
%

\documentclass[a4paper]{article}
% Typically the 'article' class is appropriate for assignments.
% And we print it on a4, so we include that as well.

\usepackage{a4wide}
% To decrease the margins and allow more text on a page.

\usepackage{graphicx}
% To deal with including pictures.

\usepackage{enumerate}
% To provide a little bit more functionality than with LaTeX's default
% enumerate environment.

\usepackage{array}
% To provide a little bit more functionality than with LaTeX's default
% array environment.

\usepackage[american]{babel}
% Use this if you want to write the document in US English. It takes care of
% (usually) proper hyphenation.
% If you want to write your answers in Dutch, please replace 'american'
% by 'dutch'.
% Note that after a change it may be that the first compilation of LaTeX
% fails. That is normal and caused by the fact that in auxiliary files
% from previous runs, there may still be a \selectlanguage{american}
% around, which is invalid if 'american' is not incorporated with babel.

\usepackage{amssymb}
% This package loads mathematical things like the fonts for the blackboard
% bold for the set of natural numbers.
\usepackage{amsmath}
% And some student asked me to include amsmath as well...

\usepackage{tikz}
\usetikzlibrary{arrows}
% The tikz package can be used to draw all kinds of diagrams.
% In this assignment it is being used for drawing the parse trees.

\usepackage[all]{xy}
% Instead of tikz you can also use xy to draw parse trees with

% Some obscure definition to create a circled node within xy.
% The definition is made by Freek Wiedijk who prefers to do his definitions
% in TeX instead of LaTeX, which explains the \def instead of \newcommand.
\def\node{*++[o][F-]}
\def\fnode{*++[o][F=]}

\newcommand{\exercise}[2]{\subsection*{Exercise #1}{#2}}
\newcommand{\exerciseenum}[2]{\subsection*{Exercise #1}{\begin{enumerate}[a)]#2\end{enumerate}}}
% We defined our own commands to make it easy to present all the
% exercises in the same style. The first one does not automatically
% start an 'enumerate' list, the second one does.
% The [2] means that our command needs two arguments.
% The #1 and the #2 indicate where we use these arguments in the
% command.
% There are several ways to have automatic numbering for the exercises,
% but here we have chosen to use a subsection for this and use manual
% numbering. This is because maybe not everyone will be able to do hand in
% all exercises.
% Note that we add the '*' to make sure that the subsection is not numbered.
% (Since we don't have a \section, the numbers for a subsection would be
% ugly like 0.1, 0.2 et cetera.
% The environment 'enumerate' automatically numbers the items in this list.
% The optional [a)] makes sure that the list will be like a), b), c) et cetera.



\newcommand{\abs}[1]{\ensuremath{\left|\, #1 \,\right|}}
\newcommand{\floor}[1]{\ensuremath{\left\lfloor\, #1 \,\right\rfloor}}
\newcommand{\ceil}[1]{\ensuremath{\left\lceil\, #1 \,\right\rceil}}
% Abbreviations for the absolute value, ceil and floor function.

\newcommand{\set}[1]{\ensuremath{\left\{{#1}\right\}}}
% This command puts curly braces around its argument, so it becomes
% a set. The \left and \right make sure that the braces grow in size
% if the contents of the set are large symbols.

\newcommand{\setbuild}[2]{\ensuremath{\set{{#1}\mid{#2}}}}
% We also introduce a shortcut for using the set builder notation.
% Do you understand what it does?

\newcommand{\seq}[1]{\ensuremath{\left\{{#1}\right\}}}
% This puts curly braces to define a sequence.
% Note that this is the same as the definition for a \set.

% And the next series of commands gives you some of the default sets
% that were in the slides.
\newcommand{\TT}{\ensuremath{\mathbb{T}}}
\newcommand{\FF}{\ensuremath{\mathbb{F}}}
\newcommand{\NN}{\ensuremath{\mathbb{N}}}
\newcommand{\NNp}{\ensuremath{\mathbb{N}^{+}}}
\newcommand{\ZZ}{\ensuremath{\mathbb{Z}}}
\newcommand{\ZZp}{\ensuremath{\mathbb{Z}^{+}}}
\newcommand{\QQ}{\ensuremath{\mathbb{Q}}}
\newcommand{\QQp}{\ensuremath{\mathbb{Q}^{+}}}
\newcommand{\RR}{\ensuremath{\mathbb{R}}}
\newcommand{\RRp}{\ensuremath{\mathbb{R}^{+}}}
\newcommand{\CC}{\ensuremath{\mathbb{C}}}

% And the next command gives a shorthand for the power set of a given set.
\newcommand{\power}[1]{\ensuremath{{\cal P}\left({#1}\right)}}

% The following two commands can be used to get an upright T or F, even
% when in math mode.
\newcommand{\Tt}{\ensuremath{\mathrm T}}
\newcommand{\Ff}{\ensuremath{\mathrm F}}

% Below is the tikz-definition that is used for the parse trees.
\tikzset{
  treenode/.style = {
    align=center,
    inner sep=0pt,
    text centered,
    font=\sffamily},
  arn_n/.style = {
    thick,
    treenode,
    circle,
    font=\sffamily\bfseries,
    draw=black,
    %text width=1.5em,
    minimum size=1.5em
    },
  arn_r/.style = {
    treenode,
    circle,
    red,
    draw=red,
    %text width=1.5em,
    minimum size=1.5em,
    very thick},
}

\title{Mathematical Structures\\Assignment 4}

% Replace the placeholders by your real name, student number and
% group (for the exercise hours)
\author{Tony Lopar \\ s1013792 \quad Group 1}

% In LaTeX everything before \begin{document} is called pre-amble.
% This is where you put all important settings. The real document
% starts after \begin{document}.
\begin{document}
\maketitle
% \maketitle makes sure that the title is shown on the first page of
% the document.


% Now we use the command we defined earlier and give it the proper two
% parameters.
% Because the second parameter is long, we put a % directly after the
% opening curly brace {. This is not needed but makes the source file
% look a bit better.
\exerciseenum{8}{%
\addtocounter{enumi}{3}
\item%d
The sentence
`No student in your class has a cat, a dog, and a ferret.'
can be expressed by the formula
% \[\forall x \in S[\neg(C(x) \land D(x) \land F(x))]\]
\[\neg \exists x \in S[C(x) \land D(x) \land F(x)]\]
%In LaTeX opening quotes should be made with ` and closing quotes with '
\item%e
The sentence
`For each of the three animals, cats, dogs, and ferrets,
    there is a student in your class who has this animal as
    a pet.'
can be expressed by the formula
\[\exists x \in S[C(x)] \land \exists y \in S[D(y)] \land \exists z \in S[F(z)]\]
}

\exerciseenum{9}{%
\addtocounter{enumi}{2}
\item%c
If the domain consists of the integers then the truth value of
`$P(2)$'
is false,
because the statement states then that $2 = 2^2$ and this is not the case since $2^2 = 4$ and $2 \neq 4$.
\addtocounter{enumi}{1}
\item%e
If the domain consists of the integers then the truth value of
`$\exists x\in \ZZ \left[ P(x) \right]$'
is true,
because if we take $x = 1$, the the statement will be $1 = 1^2$ and this is true. This means that there exists an element for which $P(x)$ is true, so the truth value of the statement is true.
% The \left[ and \right] ensure that the brackets grow in size if the
% formula is high.
% In this case simply writing '\exists x\in\ZZ [ P(x) ]' gives the
% same result.
}

\exerciseenum{10}{%
\addtocounter{enumi}{4}
\item%e
This exercise is about the sentence
`There exists a pig that can swim and catch fish.'
We start by explicitly defining our domain and predicates:
\begin{eqnarray*}
D    &:= & \mbox{all pigs}\\
S(x) &:= & \mbox{$x$ can swim}\\
C(x) &:= & \mbox{$x$ can catch fish}
\end{eqnarray*}
Now the sentence can be translated into the formula:
\[\exists x \in D[S(x) \land C(x)]\]
And the negation of this formula where there is no negation-sign to the left
of any quantifier is
\[\forall x \in D[\neg S(x) \lor \neg C(x)]\]
And if we translate this negation back into English we get the sentence
`For every pig it's the case that they can't swim or catch fish'.
}

\exercise{11}{%
The formulas
$\forall x\in D \left[ P (x) \leftrightarrow Q(x)\right]$
and
$  \forall x\in D \left[P (x)\right]
\leftrightarrow
   \forall x\in D \left[Q(x)\right]
$
are not logically equivalent,
because two formulas are only logically equivalent when they have the same truth values for all elements from all domains. In order to prove that they are not logically equivalent, we have to find that there can be an element which causes one of the two formulas to give a different truth value. \\
Let's take elements $a, b \in D$ for which $P(a)$ and $Q(a)$ are true and let $P(b)$ result true, while $Q(b)$ results false. \\
$\forall x\in D \left[ P(a) \leftrightarrow Q(a)\right]$ will be true, since $P(a)$ and $Q(a)$ are both true.
% This means that $a$ is an element in both of them.
The formula $\forall x\in D \left[P (b)\right]$ will result true for b. The other part $\forall x\in D \left[Q(b)\right]$ will be false which makes the truth values different from each other. Since the thruth values are different the biconditional statement will be false. \\
So for the two different elements a and b, the formulas have a different truth values which makes the statements not logically equivalent.
}

\exercise{12}{%
Let $P(x, y)$ be the statement ``Student $x$ follows course $y$,''
where the domain $S$ consists of all students in my class,
and the domain $C$ consists of all computer science courses
at my university.
\begin{enumerate}[a)]
\addtocounter{enumi}{1}
\item%b
The formula
$\exists x\in S\left[\forall y\in C\left[P(x, y)\right]\right]$
can be translated to the English sentence
`There is a student in my class which follows all computer science courses'
\addtocounter{enumi}{2}
\item%e
The formula
$\forall y\in C\left[\exists x\in S\left[P(x, y)\right]\right]$
can be translated to the English sentence
`For every computer science course, there is at least one student from my class that's following it.'
\end{enumerate}
}

\exercise{13}{%
Let $I (x)$ be the statement ``$x$ has an Internet connection''
and $C(x, y)$ be the statement ``$x$ and $y$ have chatted over
the Internet,'' where the domain $S$ for the variables $x$ and $y$
consists of all students in my class.
\begin{enumerate}[a)]
\addtocounter{enumi}{8}
\item%i
The sentence
`Everyone except one student in your class has an Internet connection.'
can be expressed by the formula
\[\forall x \in S[\exists y \in S[I(x) \land \neg I(y) \land x \neq y]]\]

\addtocounter{enumi}{5}
\item%o
The sentence
`There are two students in the class who combined
     have chatted with everyone in the class.'
can be expressed by the formula
% \[\exists x \in S[\exists y \in S[\forall z \in S[(C(x, z) \lor C(y, z)) \land C(x, y) \land x \neq y \land x \neq z \land y \neq z]]]\]
\[\forall x \in S[\exists y \in S[C(y, x) \land y \neq x] \lor \exists z \in S[C(z, x) \land z \neq x] \land y \neq z]\]

\end{enumerate}
}

\end{document}
