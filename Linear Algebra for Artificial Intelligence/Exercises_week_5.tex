%
% Although we try to provide a template that completely
% matches the corresponding assignment, we do expect you
% to check that you have indeed answered all questions.
%

% ALSO VERY IMPORTANT:
% This is just a template to help you with the LaTeX part of the assignment.
% So you may change it completely according to your own wishes!
%

\documentclass[a4paper]{article}
% Typically the 'article' class is appropriate for assignments.
% And we print it on a4, so we include that as well.

\usepackage{a4wide}
% To decrease the margins and allow more text on a page.

\usepackage{graphicx}
% To deal with including pictures.

\usepackage{enumerate}
% To provide a little bit more functionality than with LaTeX's default
% enumerate environment.

\usepackage{array}
% To provide a little bit more functionality than with LaTeX's default
% array environment.

\usepackage[american]{babel}
% Use this if you want to write the document in US English. It takes care of
% (usually) proper hyphenation.
% If you want to write your answers in Dutch, please replace 'american'
% by 'dutch'.
% Note that after a change it may be that the first compilation of LaTeX
% fails. That is normal and caused by the fact that in auxiliary files
% from previous runs, there may still be a \selectlanguage{american}
% around, which is invalid if 'american' is not incorporated with babel.

\usepackage{amssymb}
% This package loads mathematical things like the fonts for the blackboard
% bold for the set of natural numbers.
\usepackage{amsmath}
% And some student asked me to include amsmath as well...

\usepackage{tikz}
\usetikzlibrary{arrows}
\usetikzlibrary{positioning}
% The tikz package can be used to draw all kinds of diagrams.
% In this assignment it is being used for drawing the parse trees


\usepackage{xspace}
% xspace can be used to let LaTeX decide whether a command should be followed
% y a space or not, depending on what follows.

% Some obscure definition to create a circled node within xy.
% The definition is made by Freek Wiedijk who prefers to do his definitions
% in TeX instead of LaTeX, which explains the \def instead of \newcommand.
\def\node{*++[o][F-]}
\def\fnode{*++[o][F=]}

\usepackage{csquotes}
\usepackage{mathtools}
\usepackage{cancel}

% This command puts a `def' on top of an `='.
\newcommand{\isdef}{\ensuremath{\,\,\buildrel\rm def\over=}\,\,}

\reversemarginpar
\title{Linear Algebra for AI\\Assignment 5}

\author{Tony Lopar \\ s1013792 \\ Group 5 \quad Felicity Reddel}

\begin{document}
\maketitle

\section*{Exercise 1}
\begin{enumerate}[i)]
  \item In order to find the inverse I augmented the matrix with the identity matrix. If the original matrix has n pivots in Echelon form, there is an inverse and we may find the inverse by continuing to reduced Echelon form and take the augmented matrix.
  \[
  \left(
  \begin{array}{ccc|ccc}
  1 & -1 & 2 & 1 & 0 & 0 \\
  1 & 1 & -2 & 0 & 1 & 0 \\
  1 & 1 & 4  & 0 & 0 & 1 \\
  \end{array}
  \right)
  \xrightarrow{\text{$R_1 := R_1 + 1 \cdot R_2$}}
  \left(
  \begin{array}{ccc|ccc}
  2 & 0 & 0 & 1 & 1 & 0 \\
  1 & 1 & -2 & 0 & 1 & 0 \\
  1 & 1 & 4  & 0 & 0 & 1 \\
  \end{array}
  \right)
\]
\[
\xrightarrow{\text{$R_3 := R_3 + - \frac{1}{2} \cdot R_1$}}
\left(
\begin{array}{ccc|ccc}
2 & 0 & 0 & 1 & 1 & 0 \\
1 & 1 & -2 & 0 & 1 & 0 \\
0 & 1 & 4  & - \frac{1}{2} & - \frac{1}{2} & 1 \\
\end{array}
\right)
\xrightarrow{\text{$R_2 := R_2 + - \frac{1}{2} \cdot R_1$}}
\left(
\begin{array}{ccc|ccc}
2 & 0 & 0 & 1 & 1 & 0 \\
0 & 1 & -2 & - \frac{1}{2} & \frac{1}{2} & 0 \\
0 & 1 & 4  & - \frac{1}{2} & - \frac{1}{2} & 1 \\
\end{array}
\right)
\]
\[
\xrightarrow{\text{$R_3 := R_3 + (-1) \cdot R_2$}}
\left(
\begin{array}{ccc|ccc}
2 & 0 & 0 & 1 & 1 & 0 \\
0 & 1 & -2 & - \frac{1}{2} & \frac{1}{2} & 0 \\
0 & 0 & 6  & 0 & - 1 & 1 \\
\end{array}
\right)
\xrightarrow{\text{$R_3 := \frac{1}{6} R_3$}}
\left(
\begin{array}{ccc|ccc}
2 & 0 & 0 & 1 & 1 & 0 \\
0 & 1 & -2 & - \frac{1}{2} & \frac{1}{2} & 0 \\
0 & 0 & 1  & 0 & - \frac{1}{6} & \frac{1}{6} \\
\end{array}
\right)
\]
Now we see that the system has n pivots, we may continue to the Echelon form to find the inverse.
\[
\xrightarrow{\text{$R_2 := R_2 + 2 \cdot R_3$}}
\left(
\begin{array}{ccc|ccc}
2 & 0 & 0 & 1 & 1 & 0 \\
0 & 1 & 0 & - \frac{1}{2} & \frac{1}{6} & \frac{1}{3} \\
0 & 0 & 1  & 0 & - \frac{1}{6} & \frac{1}{6} \\
\end{array}
\right)
\xrightarrow{\text{$R_1 := \frac{1}{2} R_1$}}
\left(
\begin{array}{ccc|ccc}
1 & 0 & 0 & \frac{1}{2} & \frac{1}{2} & 0 \\
0 & 1 & 0 & - \frac{1}{2} & \frac{1}{6} & \frac{1}{3} \\
0 & 0 & 1  & 0 & - \frac{1}{6} & \frac{1}{6} \\
\end{array}
\right)
\]
Now we see that the inverse of the matrix A is as follows:
\[
A^{-1} =
\left(
\begin{array}{ccc}
\frac{1}{2} & \frac{1}{2} & 0 \\
- \frac{1}{2} & \frac{1}{6} & \frac{1}{3} \\
0 & - \frac{1}{6} & \frac{1}{6} \\
\end{array}
\right)
\]
  \item In order to find the possible inverse I will augment matrix B with the identity matrix like I did with matrix A.
  \[
  \left(
  \begin{array}{ccc|ccc}
  1 & -1 & 2 & 1 & 0 & 0 \\
  1 & 1 & -2 & 0 & 1 & 0 \\
  2 & -1 & 2  & 0 & 0 & 1 \\
  \end{array}
  \right)
  \xrightarrow{\text{$R_1 := R_1 +  1 \cdot R_2$}}
  \left(
  \begin{array}{ccc|ccc}
  2 & 0 & 0 & 1 & 1 & 0 \\
  1 & 1 & -2 & 0 & 1 & 0 \\
  2 & -1 & 2  & 0 & 0 & 1 \\
  \end{array}
  \right)
  \]
  \[
  \xrightarrow{\text{$R_1 := \frac{1}{2} R_1$}}
  \left(
  \begin{array}{ccc|ccc}
  1 & 0 & 0 & \frac{1}{2} & \frac{1}{2} & 0 \\
  1 & 1 & -2 & 0 & 1 & 0 \\
  2 & -1 & 2  & 0 & 0 & 1 \\
  \end{array}
  \right)
  \xrightarrow{\text{$R_2 := R_2 + (-1) \cdot R_1$}}
  \left(
  \begin{array}{ccc|ccc}
  1 & 0 & 0 & \frac{1}{2} & \frac{1}{2} & 0 \\
  0 & 1 & -2 & - \frac{1}{2} & \frac{1}{2} & 0 \\
  2 & -1 & 2  & 0 & 0 & 1 \\
  \end{array}
  \right)
  \]
  \[
  \xrightarrow{\text{$R_3 := R_3 + (-2) \cdot R_1$}}
  \left(
  \begin{array}{ccc|ccc}
  1 & 0 & 0 & \frac{1}{2} & \frac{1}{2} & 0 \\
  0 & 1 & -2 & - \frac{1}{2} & \frac{1}{2} & 0 \\
  0 & -1 & 2  & -1 & -1 & 1 \\
  \end{array}
  \right)
  \xrightarrow{\text{$R_3 := R_3 + 1 \cdot R_2$}}
  \left(
  \begin{array}{ccc|ccc}
  1 & 0 & 0 & \frac{1}{2} & \frac{1}{2} & 0 \\
  0 & 1 & -2 & - \frac{1}{2} & \frac{1}{2} & 0 \\
  0 & 0 & 0  & - \frac{3}{2} & - \frac{1}{2} & 1 \\
  \end{array}
  \right)
  \]
  We see that the echelon form has less than n pivots which means that there is no inverse for matrix B.
\end{enumerate}

\section*{Exercise 2}
In order to find a value for which the matrix doesn't have an inverse, we should find a value for which the matrix doesn't have a unique solution. A possible option may be to take a = 0.6, since this gives two equal columns which means the system has multiple solutions possible and thus there is no inverse. So taking a = 0.6 will give the following system:
\[
\left(
\begin{array}{cc}
0.6 & 0.6 \\
0.4 & 0.4 \\
\end{array}
\right)
\xrightarrow{\text{$R_2 := R_2 + - \frac{2}{3} \cdot R_1$}}
\left(
\begin{array}{cc}
0.6 & 0.6 \\
0 & 0 \\
\end{array}
\right)
\]
This shows that the system has less than n pivots which means that there is no inverse.

\section*{Exercise 3}
% Consider the following two bases in R2: B = {u1, u2} = {(1, 2),(3, 5)} and R2: C = {v1, v2} ={(1, −1),(1, −2)}
\begin{enumerate}
  \item The basis transformation matrix $Q = T_{C \Rightarrow B}$ can be computed as follows.
%   Given an a and b we need to find a' and b' such that:
%   $[a', b']_C = [a, b]_B$.
%   This can be witten as:
%   \[
%   a' \cdot (1, -1) + b' \cdot (1, -2) = a \cdot (1, 2) + b \cdot (3, 5)
%   \]
%   Which is written as matrix-vector multiplication like:
% \[
% \left(
% \begin{array}{cc}
% 1 & 1 \\
% -1 & -2 \\
% \end{array}
% \right)
% \cdot
% \left(
% \begin{array}{c}
% a' \\
% b' \\
% \end{array}
% \right)
% =
% \left(
% \begin{array}{cc}
% 1 & 3 \\
% 2 & 5 \\
% \end{array}
% \right)
% \cdot
% \left(
% \begin{array}{c}
% a \\
% b \\
% \end{array}
% \right)
% \]
% This can be solved using matrix multiplication:
% \[
% \left(
% \begin{array}{c}
% a' \\
% b' \\
% \end{array}
% \right)
% =
% \left(
% \begin{array}{cc}
% 1 & 1 \\
% -1 & -2 \\
% \end{array}
% \right)^{-1}
% \cdot
% \left(
% \begin{array}{cc}
% 1 & 3 \\
% 2 & 5 \\
% \end{array}
% \right)
% \cdot
% \left(
% \begin{array}{c}
% a \\
% b \\
% \end{array}
% \right)
% \]
If we put the vectors of bases B and C into matrices, we may compute the transformation matrix with the following computation:
\[
Q =
\left(
\begin{array}{cc}
1 & 3 \\
2 & 5 \\
\end{array}
\right)^{-1}
\cdot
\left(
\begin{array}{cc}
1 & 1 \\
-1 & -2 \\
\end{array}
\right)
\]
Using the solution for 2x2 matrices from the slides, we find the following inverse for B.
\[
A_B^{-1} =
\frac{1}{1 \cdot 5 - 3 \cdot 2} \cdot
\left(
\begin{array}{cc}
5 & -3 \\
-2 & 1 \\
\end{array}
\right)
=
-1 \cdot
\left(
\begin{array}{cc}
5 & -3 \\
-2 & 1 \\
\end{array}
\right)
=
\left(
\begin{array}{cc}
-5 & 3 \\
2 & -1 \\
\end{array}
\right)
\]
This gives that P is defined as the following matrix:
\begin{align*}
Q &=
\left(
\begin{array}{cc}
-5 & 3 \\
2 & -1 \\
\end{array}
\right)
\cdot
\left(
\begin{array}{cc}
1 & 1 \\
-1 & -2 \\
\end{array}
\right) \\
&=
\left(
\begin{array}{cc}
-8 & -11 \\
3 & 4 \\
\end{array}
\right)
\end{align*}

  \item The basis transformation matrix $P = T_{B \Rightarrow C}$ can be computed by multiplying the inverse of B with matrix C.
  \[
  P =
  \left(
  \begin{array}{cc}
  1 & 1 \\
  -1 & -2 \\
  \end{array}
  \right)^{-1}
  \cdot
  \left(
  \begin{array}{cc}
  1 & 3 \\
  2 & 5 \\
  \end{array}
  \right)
  \]
  Using the solution for 2x2 matrices from the slides, we find the following inverse for C.
  \[
  A_C^{-1} =
  \frac{1}{1 \cdot -2 - 1 \cdot -1} \cdot
  \left(
  \begin{array}{cc}
  -2 & -1 \\
  1 & 1 \\
  \end{array}
  \right)
  =
  -1 \cdot
  \left(
  \begin{array}{cc}
  -2 & -1 \\
  1 & 1 \\
  \end{array}
  \right)
  =
  \left(
  \begin{array}{cc}
  2 & 1 \\
  -1 & -1 \\
  \end{array}
  \right)
  \]
  This gives that Q is defined as the following matrix:
  \begin{align*}
  P &=
  \left(
  \begin{array}{cc}
  2 & 1 \\
  -1 & -1 \\
  \end{array}
  \right)
  \cdot
  \left(
  \begin{array}{cc}
  1 & 3 \\
  2 & 5 \\
  \end{array}
  \right) \\
  &=
  \left(
  \begin{array}{cc}
  4 & 11 \\
  -3 & -8 \\
  \end{array}
  \right)
  \end{align*}
  \item First, I will compute $Q \cdot P$:
  \begin{align*}
  \left(
  \begin{array}{cc}
  -8 & -11 \\
  3 & 4 \\
  \end{array}
  \right)
  \cdot
  \left(
  \begin{array}{cc}
  4 & 11 \\
  -3 & -8 \\
  \end{array}
  \right)
  &=
  \left(
  \begin{array}{cc}
  1 & 0 \\
  0 & 1 \\
  \end{array}
  \right)
  \end{align*}
  Now I will compute $P \cdot Q$:
  \begin{align*}
  \left(
  \begin{array}{cc}
  4 & 11 \\
  -3 & -8 \\
  \end{array}
  \right)
  \cdot
  \left(
  \begin{array}{cc}
  -8 & -11 \\
  3 & 4 \\
  \end{array}
  \right)
  &=
  \left(
  \begin{array}{cc}
  1 & 0 \\
  0 & 1 \\
  \end{array}
  \right)
  \end{align*}
  \item The matrix $A_C$ can be computed with the matrices given in the exercise and the computed matrices P and Q. $A_C = T_{B \Rightarrow C} \cdot A_B \cdot T_{C \Rightarrow B}$.
  \begin{align*}
  A_C &= P \cdot A_B \cdot Q \\
  &=
  \left(
  \begin{array}{cc}
  4 & 11 \\
  -3 & -8 \\
  \end{array}
  \right)
  \cdot
  \left(
  \begin{array}{cc}
  35 & 88 \\
  -12 & -30 \\
  \end{array}
  \right)
  \cdot
  \left(
  \begin{array}{cc}
  -8 & -11 \\
  3 & 4 \\
  \end{array}
  \right) \\
  &=
  \left(
  \begin{array}{cc}
  8 & 22 \\
  -9 & -24 \\
  \end{array}
  \right)
  \cdot
  \left(
  \begin{array}{cc}
  -8 & -11 \\
  3 & 4 \\
  \end{array}
  \right) \\
  &=
  \left(
  \begin{array}{cc}
  2 & 0 \\
  0 & 3 \\
  \end{array}
  \right)
  \end{align*}
\end{enumerate}

\section*{Exercise 4}
If we first map and then basis transform this will give the following situation:
\[
x_B \xrightarrow{A_B} y_B \xrightarrow{T_{B \Rightarrow C}} y_C
\]
This shows us that $y_B = A_B \cdot x_B$ and that $y_C = T_{B \Rightarrow C} \cdot y_B = T_{B \Rightarrow C} \cdot A_B \cdot x_B$.

Next, we will first transform and map the vector afterwards. This will give the following transformations:
\[
x_B \xrightarrow{T_{B \Rightarrow C}} x_C \xrightarrow{A_C} y_C
\]

This shows us that $x_C = T_{B \Rightarrow C} \cdot x_B$ and $y_C = A_C \cdot x_C = A_C \cdot T_{B \Rightarrow C} \cdot x_B$.
\newline
\newline
Combining these two solutions gives us that:
\begin{align*}
  y_C &= y_C \\
  A_C \cdot T_{B \Rightarrow C} &= T_{B \Rightarrow C} \cdot A_B
  \end{align*}
  Multiplying by the inverse of $T_{B \Rightarrow C}$ to get only $A_C$ on the left of the equation:
  \begin{align*}
  A_C \cdot T_{B \Rightarrow C} \cdot {T_{B \Rightarrow C}}^{-1} &= T_{B \Rightarrow C} \cdot A_B \cdot {T_{B \Rightarrow C}}^{-1} \\
  A_C \cdot \cancel{T_{B \Rightarrow C} \cdot {T_{B \Rightarrow C}}^{-1}} &= T_{B \Rightarrow C} \cdot A_B \cdot T_{C \Rightarrow B} \\
  A_C &= T_{B \Rightarrow C} \cdot A_B \cdot T_{C \Rightarrow B} \\
\end{align*}

\end{document}
