%
% Although we try to provide a template that completely
% matches the corresponding assignment, we do expect you
% to check that you have indeed answered all questions.
%

% ALSO VERY IMPORTANT:
% This is just a template to help you with the LaTeX part of the assignment.
% So you may change it completely according to your own wishes!
%

\documentclass[a4paper]{article}
% Typically the 'article' class is appropriate for assignments.
% And we print it on a4, so we include that as well.

\usepackage{a4wide}
% To decrease the margins and allow more text on a page.

\usepackage{graphicx}
% To deal with including pictures.

\usepackage{enumerate}
% To provide a little bit more functionality than with LaTeX's default
% enumerate environment.

\usepackage{array}
% To provide a little bit more functionality than with LaTeX's default
% array environment.

\usepackage[american]{babel}
% Use this if you want to write the document in US English. It takes care of
% (usually) proper hyphenation.
% If you want to write your answers in Dutch, please replace 'american'
% by 'dutch'.
% Note that after a change it may be that the first compilation of LaTeX
% fails. That is normal and caused by the fact that in auxiliary files
% from previous runs, there may still be a \selectlanguage{american}
% around, which is invalid if 'american' is not incorporated with babel.

\usepackage{amssymb}
% This package loads mathematical things like the fonts for the blackboard
% bold for the set of natural numbers.
\usepackage{amsmath}
% And some student asked me to include amsmath as well...

\usepackage{tikz}
\usetikzlibrary{arrows}
\usetikzlibrary{positioning}
% The tikz package can be used to draw all kinds of diagrams.
% In this assignment it is being used for drawing the parse trees


\usepackage{xspace}
% xspace can be used to let LaTeX decide whether a command should be followed
% y a space or not, depending on what follows.

% Some obscure definition to create a circled node within xy.
% The definition is made by Freek Wiedijk who prefers to do his definitions
% in TeX instead of LaTeX, which explains the \def instead of \newcommand.
\def\node{*++[o][F-]}
\def\fnode{*++[o][F=]}

\usepackage{csquotes}
\usepackage{mathtools}
\usepackage{cancel}

% This command puts a `def' on top of an `='.
\newcommand{\isdef}{\ensuremath{\,\,\buildrel\rm def\over=}\,\,}

\reversemarginpar
\title{Linear Algebra for AI\\Assignment 6}

\author{Tony Lopar \\ s1013792 \\ Group 5 \quad Felicity Reddel}

\begin{document}
\maketitle

\section*{Exercise 1}
In order to find the determinant I will first reduce the matrix to Echelon form and then multiply the pivots with eachother. In order to find the original determinant, I will inverse all the modifications on the matrix on the value of the determinant to remove all influences of the modifications from the determinant.
\[
\left(
\begin{array}{rrrr}
2 & 5 & -3 & -2 \\
-2 & -3 & 2 & -5 \\
1 & 3 & -2 & 0 \\
-1 & -6 & 4 & 0 \\
\end{array}
\right)
\xrightarrow{R_2 := R_2 + 1 \cdot R_1}
\left(
\begin{array}{rrrr}
2 & 5 & -3 & -2 \\
0 & 2 & -1 & -7 \\
1 & 3 & -2 & 0 \\
-1 & -6 & 4 & 0 \\
\end{array}
\right)
\xrightarrow{R_3 := R_3 + (- \frac{1}{2}) \cdot R_1}
\left(
\begin{array}{rrrr}
2 & 5 & -3 & -2 \\
0 & 2 & -1 & -7 \\
0 & \frac{1}{2} & - \frac{1}{2} & 1 \\
-1 & -6 & 4 & 0 \\
\end{array}
\right)
\]
\[
\xrightarrow{R_4 := R_4 + \frac{1}{2} \cdot R_1}
\left(
\begin{array}{rrrr}
2 & 5 & -3 & -2 \\
0 & 2 & -1 & -7 \\
0 & \frac{1}{2} & - \frac{1}{2} & 1 \\
0 & - \frac{7}{2} & \frac{5}{2} & -1 \\
\end{array}
\right)
\xrightarrow{R_4 := R_4 + 7 \cdot R_3}
\left(
\begin{array}{rrrr}
2 & 5 & -3 & -2 \\
0 & 2 & -1 & -7 \\
0 & \frac{1}{2} & - \frac{1}{2} & 1 \\
0 & 0 & -1 & 6 \\
\end{array}
\right)
\]
\[
\xrightarrow{R_3 := R_3 + (-\frac{1}{4}) \cdot R_2}
\left(
\begin{array}{rrrr}
2 & 5 & -3 & -2 \\
0 & 2 & -1 & -7 \\
0 & 0 & - \frac{1}{4} & \frac{11}{4} \\
0 & 0 & -1 & 6 \\
\end{array}
\right)
\xrightarrow{R_4 := R_4 + (-4) \cdot R_3}
\left(
\begin{array}{rrrr}
2 & 5 & -3 & -2 \\
0 & 2 & -1 & -7 \\
0 & 0 & - \frac{1}{4} & \frac{11}{4} \\
0 & 0 & 0 & -5 \\
\end{array}
\right)
\]
Multiplying the pivots gives the following:
\[det(A) = 2 \cdot 2 \cdot - \frac{1}{4} \cdot -5 = 5\]
This determinant has not been influenced by the modifications on the matrix, since no rows are swapped or multiplied by itself. The only modifications are a adding another row to the rows. This modification doesn't change the determinant which means the determinant is equal to the current multiplication of pivots.
% Reduce to echelon and find the determinant by multiplying. Then tranform the inverse transformations that were done on the matrix in the opposite way to get the original det?

\section*{Exercise 2}
\begin{enumerate}
  \item The characteristic polynomial can be obtained by computing $A - \lambda \cdot I$. So first we need to compute:
  \[
  A - \lambda \cdot I =
  \left(
  \begin{array}{rrr}
  1 & -3 & 4 \\
  4 & -7 & 8 \\
  6 & -7 & 7 \\
  \end{array}
  \right)
  -
  \left(
  \begin{array}{rrr}
  \lambda & 0 & 0 \\
  0 & \lambda & 0 \\
  0 & 0 & \lambda \\
  \end{array}
  \right)
  =
  \left(
  \begin{array}{rrr}
  1 - \lambda & -3 & 4 \\
  4 & -7 - \lambda & 8 \\
  6 & -7 & 7 - \lambda \\
  \end{array}
  \right)
  \]
  This gives the following characteristic polynomial:
  \[
  \left|
  \begin{array}{rrr}
  1 - \lambda & -3 & 4 \\
  4 & -7 - \lambda & 8 \\
  6 & -7 & 7 - \lambda \\
  \end{array}
  \right|
  \]
  \item The eigenvalues may be computed using the characteristic polynomial from the previous question.
  \begin{align*}
  det(A - \lambda \cdot I) &=
  (1 - \lambda)
  \left|
  \begin{array}{rr}
  -7 - \lambda & 8 \\
  -7 & 7 - \lambda \\
  \end{array}
  \right|
  - 4
  \left|
  \begin{array}{rr}
  -3 & 4 \\
  -7 & 7 - \lambda \\
  \end{array}
  \right|
  + 6
  \left|
  \begin{array}{rr}
  -3 & 4 \\
  -7 - \lambda & 8 \\
  \end{array}
  \right| \\
  &= (1 - \lambda)((-7-\lambda)(7 - \lambda) + 56) - 4 (-3(7 - \lambda) + 28) + 6(-24 - 4(-7 - \lambda)) \\
&= (1 - \lambda)((-49 + 7 \lambda - 7 \lambda + \lambda^2) + 56) - 4(-21 + 3 \lambda + 28) + 6(-24 + 28 + 4 \lambda)\\
&= (1 - \lambda)(\lambda^2 + 7) - 4(3 \lambda + 7) + 6(4 + 4 \lambda)\\
&= - \lambda^3 + \lambda^2 - 7 \lambda + 7 - 12 \lambda - 28 + 24 + 24 \lambda \\
&= - \lambda^3 + \lambda^2 + 5 \lambda + 3
  \end{align*}
  To find the eigenvalues we should find the values for $\lambda$. These can be found as follows:
  \begin{align*}
    - \lambda^3 + \lambda^2 + 5 \lambda + 3 &= 0 \\
    (\lambda + 1)(- \lambda^2 + 2 \lambda + 3) &= 0 \\
    (\lambda + 1) \cdot -1(\lambda^2 - 2 \lambda - 3) &= 0 \\
    (\lambda + 1) \cdot -1((\lambda - 3) (\lambda + 1)) &= 0 \\
    (\lambda + 1) (-\lambda + 3) (-\lambda - 1) &= 0 \\
    \lambda = - 1 \lor \lambda = 3 \lor \lambda = - 1
  \end{align*}
  We see that the eigenvalues of the matrix are $\lambda = -1$ and $\lambda = 3$.
  % \begin{align*}
  %   0 &= - \lambda^3 + \lambda^2 + 5 \lambda + 3 \\
  %   3 &= \lambda^3 - \lambda^2 - 5 \lambda \\
  %    &= \lambda (\lambda^2 - \lambda - 5) \\
  %   % &= (\lambda - 1)(a \lambda^2 + b \lambda + c) \\
  %   % &= a \lambda^3 + (b - a) \lambda^2 + (c - b) \lambda - c \\
  %   % &= - 1 \lambda^3 + (0 + 1) \lambda^2 + (-3 - b) \lambda + 3 \\
  % \end{align*}
  % This gives us that $\lambda = 3$ or $(\lambda^2 - \lambda - 5) = 3$. We can solve the second equation with the abc-formula.
  % \begin{align*}
  %   \lambda^2 - \lambda - 5 &= 3 \\
  %   \lambda^2 - \lambda - 8 &= 0 \\
  %   s_{1,2} + 3
  %   % &= \frac{-b \pm \sqrt{b^2 - 4ac}}{2a} \\
  %   &= \frac{1 \pm \sqrt{(-1)^2 - 4 \cdot 1 \cdot - 5}}{2 \cdot 1} \\
  %   &= \frac{1 \pm \sqrt{1 + 20}}{2} \\
  %   &= \frac{1 \pm \sqrt{21}}{2} \\
  %   &=
  % \end{align*}
  \item The eigenvector may be computed by using the eigenvalues found in the previous question. We first take $\lambda = -1$ which gives:
  \begin{align*}
    (A - - 1 \cdot I) &=
    \left(
    \begin{array}{rrr}
    1 + 1 & -3 & 4 \\
    4 & -7 + 1 & 8 \\
    6 & -7 & 7 + 1 \\
    \end{array}
    \right) \\
    &=
    \left(
    \begin{array}{rrr}
    2 & -3 & 4 \\
    4 & -6 & 8 \\
    6 & -7 & 8 \\
    \end{array}
    \right)
  \end{align*}
  In order to find the vector we should reduce this matrix to echelon form.
  \[
  \left(
  \begin{array}{rrr}
  2 & -3 & 4 \\
  4 & -6 & 8 \\
  6 & -7 & 8 \\
  \end{array}
  \right)
  \xrightarrow{R_2 := R_2 + (-2) \cdot R_1}
  \left(
  \begin{array}{rrr}
  2 & -3 & 4 \\
  0 & 0 & 0 \\
  6 & -7 & 8 \\
  \end{array}
  \right)
  \xrightarrow{R_2 \leftrightarrow R_3}
  \left(
  \begin{array}{rrr}
  2 & -3 & 4 \\
  6 & -7 & 8 \\
  0 & 0 & 0 \\
  \end{array}
  \right)
  \]
  \[
  \xrightarrow{R_2 := R_2 + (-3) R_1}
  \left(
  \begin{array}{rrr}
  2 & -3 & 4 \\
  0 & 2 & -4 \\
  0 & 0 & 0 \\
  \end{array}
  \right)
  \xrightarrow{R_2 := \frac{1}{2} R_2}
  \left(
  \begin{array}{rrr}
  2 & -3 & 4 \\
  0 & 1 & -2 \\
  0 & 0 & 0 \\
  \end{array}
  \right)
  \]
  \[
  \xrightarrow{R_1 := R_1 + 3 \cdot R_2}
  \left(
  \begin{array}{rrr}
  2 & 0 & -2 \\
  0 & 1 & -2 \\
  0 & 0 & 0 \\
  \end{array}
  \right)
  \xrightarrow{R_1 := \frac{1}{2} R_1}
  \left(
  \begin{array}{rrr}
  1 & 0 & -1 \\
  0 & 1 & -2 \\
  0 & 0 & 0 \\
  \end{array}
  \right)
  \]
  This gives the following system of equations:
  \begin{align*}
    x - z &= 0 \\
    x &= z \\
    y - 2z &= 0 \\
    y &= 2z \\
  \end{align*}
  If we take $z = 1$ then we get the following eigenvector:
  \[
  \left(
  \begin{array}{r}
  1 \\
  2 \\
  1 \\
  \end{array}
  \right)
  \]

  Now we can compute the eigenvector for $\lambda = 3$.
  \begin{align*}
    (A - 3 \cdot I) &=
    \left(
    \begin{array}{rrr}
    1 - 3 & -3 & 4 \\
    4 & -7 - 3 & 8 \\
    6 & -7 & 7 - 3 \\
    \end{array}
    \right) \\
    &=
    \left(
    \begin{array}{rrr}
    -2 & -3 & 4 \\
    4 & -10 & 8 \\
    6 & -7 & 4 \\
    \end{array}
    \right)
  \end{align*}
  In order to find the vector we should reduce this matrix to echelon form.
  \[
  \left(
  \begin{array}{rrr}
  -2 & -3 & 4 \\
  4 & -10 & 8 \\
  6 & -7 & 4 \\
  \end{array}
  \right)
  \xrightarrow{R2 := R_2 + 2 \cdot R_1}
  \left(
  \begin{array}{rrr}
  -2 & -3 & 4 \\
  0 & -16 & 16 \\
  6 & -7 & 4 \\
  \end{array}
  \right)
  \xrightarrow{R3 := R_3 + 3 \cdot R_1}
  \left(
  \begin{array}{rrr}
  -2 & -3 & 4 \\
  0 & -16 & 16 \\
  0 & -16 & 16 \\
  \end{array}
  \right)
  \]
  \[
  \xrightarrow{R3 := R_3 + (-1) \cdot R_2}
  \left(
  \begin{array}{rrr}
  -2 & -3 & 4 \\
  0 & -16 & 16 \\
  0 & 0 & 0 \\
  \end{array}
  \right)
  \xrightarrow{R2 := -\frac{1}{16} R_2}
  \left(
  \begin{array}{rrr}
  -2 & -3 & 4 \\
  0 & 1 & -1 \\
  0 & 0 & 0 \\
  \end{array}
  \right)
  \]
  \[
  \xrightarrow{R_1 := R_1 + 3 \cdot R_2}
  \left(
  \begin{array}{rrr}
  -2 & 0 & 1 \\
  0 & 1 & -1 \\
  0 & 0 & 0 \\
  \end{array}
  \right)
  \xrightarrow{R_1 := - \frac{1}{2} \cdot R_1}
  \left(
  \begin{array}{rrr}
  1 & 0 & - \frac{1}{2} \\
  0 & 1 & -1 \\
  0 & 0 & 0 \\
  \end{array}
  \right)
  \]
  This gives us the following system of equations:
  \begin{align*}
    x - \frac{1}{2}z &= 0 \\
    x &= \frac{1}{2}z \\
    y - z &= 0 \\
    y &= z
  \end{align*}
  If we take $z = 2$, then we get the following eigenvector:
  \[
  \left(
  \begin{array}{r}
  1 \\
  2 \\
  2 \\
  \end{array}
  \right)
  \]

\end{enumerate}

\section*{Exercise 3}
\begin{enumerate}
  \item First, I will show that if we swap the first and second row that $det(B) = - det(A)$.
  \[
  A =
  \left(
  \begin{array}{cc}
  a & b \\
  c & d \\
  \end{array}
  \right)
  \enspace
  B =
  \left(
  \begin{array}{cc}
  c & d \\
  a & b \\
  \end{array}
  \right)
  \]
  The determinant of matrix A is $ad - bc$. The determinant of matrix B is $cb - da$. We can do the following transformations to express the $det(B)$ in the form of $det(A)$.
  \begin{align*}
    det(B) &= cb - da \\
    &= bc - ad \\
    &= - ad + bc \\
    &= - (ad - bc) \\
    &= - (det(A))
  \end{align*}
\item Next, I will show that if we multiply row $R_2$ with p the determinant becomes $pR_2$ where I choose $p = 2$.
\[
A =
\left(
\begin{array}{cc}
a & b \\
c & d \\
\end{array}
\right)
\enspace
C =
\left(
\begin{array}{cc}
a & b \\
2c & 2d \\
\end{array}
\right)
\]
The determinant of matrix A is $ad - bc$. The determinant of matrix C is $a2d - b2c$. We can do the following transformations to express $det(C)$ in the form of det(A).
\begin{align*}
  det(C) &= a2d - b2c \\
  &= 2ad - 2bc \\
  &= 2(ad - bc) \\
  &= 2(det(A)) \\
  &= p(det(A))
\end{align*}
\item Finally, I will show that the determinant doesn't change when we add $pR_1$ to $R_2$. I choose 1 as value for p.
\[
A =
\left(
\begin{array}{cc}
a & b \\
c & d \\
\end{array}
\right)
\enspace
D =
\left(
\begin{array}{cc}
a & b \\
c + a & d + b \\
\end{array}
\right)
\]
The determinant of matrix A is $ad - bc$. The determinant of matrix D is $a(d + b) - b(c + a)$. We can do the following transformations to express $det(D)$ in the form of det(A).
\begin{align*}
  det(D) &=
  a(d + b) - b(c + a) \\ &= ad + ab - bc - ba \\
  &= ad - bc + ba - ba \\
  &= ad - bc \\
  &= det(A)
\end{align*}
% 2. Let matrix C result from A via R2 := pR2 (for p ∈ R).
% Show that det(C) = p det(A)
% 3. Let matrix D result from A via R2 := R2 + pR1 (for p ∈ R).
% Show that det(D) = det(A)

\end{enumerate}

\section*{Exercise 4}
% \[
% A =
% \left(
% \begin{array}{rrr}
% -7 & -16 & 4 \\
% 6 & 13 & -2 \\
% 12 & 16 & 1 \\
% \end{array}
% \right)
% \]
The eigenvalue for $v_1$ may be computed by multiplying the original matrix the eigenvector. This should give a matrix which is a multiple of the eigenvector. This multiple is the corresponding eigenvalue.
\[
\left(
\begin{array}{rrr}
-7 & -16 & 4 \\
6 & 13 & -2 \\
12 & 16 & 1 \\
\end{array}
\right)
\cdot
\left(
\begin{array}{r}
-2 \\
1 \\
2 \\
\end{array}
\right)
=
\left(
\begin{array}{r}
6 \\
-3 \\
-6 \\
\end{array}
\right)
=
-3 \cdot
\left(
\begin{array}{r}
-2 \\
1 \\
2 \\
\end{array}
\right)
\]
In the computation above we see that the corresponding eigenvalue $\lambda = -3$ for the eigenvector $v_1$.

% v1 = (−2, 1, 2). \lambda = -3


The eigenvector for the eigenvalue $\lambda_2 = 5$ can be computed as follows:
\begin{align*}
  (A - 5 \cdot I) &= \left(
  \begin{array}{rrr}
  -7 - 5 & -16 & 4 \\
  6 & 13 - 5 & -2 \\
  12 & 16 & 1 - 5 \\
  \end{array}
  \right) \\
  &= \left(
  \begin{array}{rrr}
  -12 & -16 & 4 \\
  6 & 8 & -2 \\
  12 & 16 & -4 \\
  \end{array}
  \right) \\
\end{align*}
In order to find the eigenvector we should reduce the matrix to Echelon form:
\[
\left(
\begin{array}{rrr}
-12 & -16 & 4 \\
6 & 8 & -2 \\
12 & 16 & -4 \\
\end{array}
\right)
\xrightarrow{R_3 := R_3 + 1 \cdot R_1}
\left(
\begin{array}{rrr}
-12 & -16 & 4 \\
6 & 8 & -2 \\
0 & 0 & 0 \\
\end{array}
\right)
\xrightarrow{R_2 := R_2 + \frac{1}{2} \cdot R_1}
\left(
\begin{array}{rrr}
-12 & -16 & 4 \\
0 & 0 & 0 \\
0 & 0 & 0 \\
\end{array}
\right)
\]
\[
\xrightarrow{R_1 := - \frac{1}{4} R_1}
\left(
\begin{array}{rrr}
3 & 4 & -1 \\
0 & 0 & 0 \\
0 & 0 & 0 \\
\end{array}
\right)
\xrightarrow{R_1 := \frac{1}{3} R_1}
\left(
\begin{array}{rrr}
1 & \frac{4}{3} & - \frac{1}{3} \\
0 & 0 & 0 \\
0 & 0 & 0 \\
\end{array}
\right)
\]
This gives us the following equation:
\begin{align*}
  x + \frac{4}{3}y - \frac{1}{3}z &= 0 \\
  x &= - \frac{4}{3}y + \frac{1}{3}z
\end{align*}
When we take $y = 1$ and $z = 1$, then we get the following eigenvectors:
\[
\left(
\begin{array}{r}
- \frac{4}{3} \\
1 \\
0 \\
\end{array}
\right)
\text{and}
\left(
\begin{array}{r}
\frac{1}{3} \\
0 \\
1 \\
\end{array}
\right)
\]
\end{document}
