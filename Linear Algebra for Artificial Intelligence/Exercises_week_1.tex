%
% Although we try to provide a template that completely
% matches the corresponding assignment, we do expect you
% to check that you have indeed answered all questions.
%

% ALSO VERY IMPORTANT:
% This is just a template to help you with the LaTeX part of the assignment.
% So you may change it completely according to your own wishes!
%

\documentclass[a4paper]{article}
% Typically the 'article' class is appropriate for assignments.
% And we print it on a4, so we include that as well.

\usepackage{a4wide}
% To decrease the margins and allow more text on a page.

\usepackage{graphicx}
% To deal with including pictures.

\usepackage{enumerate}
% To provide a little bit more functionality than with LaTeX's default
% enumerate environment.

\usepackage{array}
% To provide a little bit more functionality than with LaTeX's default
% array environment.

\usepackage[american]{babel}
% Use this if you want to write the document in US English. It takes care of
% (usually) proper hyphenation.
% If you want to write your answers in Dutch, please replace 'american'
% by 'dutch'.
% Note that after a change it may be that the first compilation of LaTeX
% fails. That is normal and caused by the fact that in auxiliary files
% from previous runs, there may still be a \selectlanguage{american}
% around, which is invalid if 'american' is not incorporated with babel.

\usepackage{amssymb}
% This package loads mathematical things like the fonts for the blackboard
% bold for the set of natural numbers.
\usepackage{amsmath}
% And some student asked me to include amsmath as well...

\usepackage{tikz}
\usetikzlibrary{arrows}
\usetikzlibrary{positioning}
% The tikz package can be used to draw all kinds of diagrams.
% In this assignment it is being used for drawing the parse trees


\usepackage{xspace}
% xspace can be used to let LaTeX decide whether a command should be followed
% y a space or not, depending on what follows.

% Some obscure definition to create a circled node within xy.
% The definition is made by Freek Wiedijk who prefers to do his definitions
% in TeX instead of LaTeX, which explains the \def instead of \newcommand.
\def\node{*++[o][F-]}
\def\fnode{*++[o][F=]}

\usepackage{csquotes}
\usepackage{mathtools}

% This command puts a `def' on top of an `='.
\newcommand{\isdef}{\ensuremath{\,\,\buildrel\rm def\over=}\,\,}

\reversemarginpar
\title{Linear Algebra for AI\\Assignment 1}

% Replace the placeholders by your real name, student number and
% group (for the exercise hours)
\author{Tony Lopar \\ s1013792 \\ Group 5 \quad Felicity Reddel}

% In LaTeX everything before \begin{document} is called pre-amble.
% This is where you put all important settings. The real document
% starts after \begin{document}.
\begin{document}
\maketitle

\section*{Exercise 1}
We are having the following linear equations: \\
\begin{align*}
x - y + z &= -3 \\
-3x + 4y - z &= 2 \\
x - 3y - 2z &= 7
\end{align*}

When we put the values of the equations into a coefficient matrix, it would look like this: \\
\[
\left(
\begin{array}{ccc}
1 & -1 & 1   \\
-3 & 4 & -1 \\
1 & -3 & -2    \\
\end{array}
\right)
\]

The augmented matrix would be:
\[
\left(
\begin{array}{ccc|c}
1 & -1 & 1 & -3   \\
-3 & 4 & -1 & 2  \\
1 & -3 & -2 & 7   \\
\end{array}
\right)
\]

In order to solve the equations we should try to get the matrix to (reduced) echelon form.
\[
\left(
\begin{array}{ccc|c}
1 & -1 & 1 & -3   \\
-3 & 4 & -1 & 2  \\
1 & -3 & -2 & 7   \\
\end{array}
\right)
\xrightarrow{\text{$R_3 := R_3 - R_1$}}
\left(
\begin{array}{ccc|c}
1 & -1 & 1 & -3   \\
-3 & 4 & -1 & 2  \\
0 & -2 & -3 & 10   \\
\end{array}
\right)
\xrightarrow{\text{$R_2 := R_2 + 3 \cdot R_1$}}
\left(
\begin{array}{ccc|c}
1 & -1 & 1 & -3   \\
0 & 1 & 2 & -7  \\
0 & -2 & -3 & 10   \\
\end{array}
\right)
\]
\[
\xrightarrow{\text{$R_3 := R_3 + 2 \cdot R_2$}}
\left(
\begin{array}{ccc|c}
1 & -1 & 1 & -3   \\
0 & 1 & 2 & -7  \\
0 & 0 & 1 & -4   \\
\end{array}
\right)
\xrightarrow{\text{$R_2 := R_2 + -(2 \cdot R_3)$}}
\left(
\begin{array}{ccc|c}
1 & -1 & 1 & -3   \\
0 & 1 & 0 & 1 \\
0 & 0 & 1 & -4   \\
\end{array}
\right)
\xrightarrow{\text{$R_1 := R_1 + R_2$}}
\left(
\begin{array}{ccc|c}
1 & 0 & 1 & -2   \\
0 & 1 & 0 & 1 \\
0 & 0 & 1 & -4   \\
\end{array}
\right)
\]
\[
\xrightarrow{\text{$R_1 := R_1 + -(R_3)$}}
\left(
\begin{array}{ccc|c}
1 & 0 & 0 & 2   \\
0 & 1 & 0 & 1 \\
0 & 0 & 1 & -4   \\
\end{array}
\right)
\]

The result of the process is the following reduced echelon matrix:

\[
\left(
\begin{array}{ccc|c}
1 & 0 & 0 & 2   \\
0 & 1 & 0 & 1 \\
0 & 0 & 1 & -4   \\
\end{array}
\right)
\]
The reduced echelon matrix shows us that there is a unique solution, so the equations are consistent.

When we fill in the results of the reduced echelon matrix in the original equations, we see that they hold.\\
\begin{align*}
1 \cdot 2 + (-1) \cdot 1 + 1 \cdot -4 &= -3 \\
-3 \cdot 2 + 4 \cdot 1 + -1 \cdot (-4) &= 2 \\
1 \cdot 2 + (-3) \cdot 1 - 2 \cdot  (-4) &= 7
\end{align*}

\section*{Exercise 2}
We are having the following linear equations: \\
\begin{align*}
x_1 - 3x_2 - 6x_3 &= 2 \\
3x_1 - 8x_2 - 17x_3 &= -1 \\
x_1 - 4x_2 - 7x_3 &= 10
\end{align*}

When we put the values of the equations into a coefficient matrix, it would look like this: \\
\[
\left(
\begin{array}{ccc}
1 & -3 & -6   \\
3 & -8 & -17 \\
1 & -4 & -7    \\
\end{array}
\right)
\]

The augmented matrix would be:
\[
\left(
\begin{array}{ccc|c}
1 & -3 & -6 & 2   \\
3 & -8 & -17 & -1 \\
1 & -4 & -7 & 10   \\
\end{array}
\right)
\]

In order to solve the equations we should try to get the matrix to (reduced) echelon form.
\[
\left(
\begin{array}{ccc|c}
1 & -3 & -6 & 2   \\
3 & -8 & -17 & -1 \\
1 & -4 & -7 & 10   \\
\end{array}
\right)
\xrightarrow{\text{$R_2 := R_2 + (-3) \cdot R_1$}}
\left(
\begin{array}{ccc|c}
1 & -3 & -6 & 2   \\
0 & 1 & 1 & -7 \\
1 & -4 & -7 & 10   \\
\end{array}
\right)
\xrightarrow{\text{$R_3 := R_3 + (-1) \cdot R_1$}}
\left(
\begin{array}{ccc|c}
1 & -3 & -6 & 2   \\
0 & 1 & 1 & -7 \\
0 & -1 & -1 & 8   \\
\end{array}
\right)
\]
\[
\xrightarrow{\text{$R_3 := R_3 + 1 \cdot R_2 $}}
\left(
\begin{array}{ccc|c}
1 & -3 & -6 & 2   \\
0 & 1 & 1 & -7 \\
0 & 0 & 0 & 1   \\
\end{array}
\right)
\xrightarrow{\text{$R_2 := R_2 + 7 \cdot R_3 $}}
\left(
\begin{array}{ccc|c}
1 & -3 & -6 & 2   \\
0 & 1 & 1 & 0 \\
0 & 0 & 0 & 1   \\
\end{array}
\right)
\]
\[
\xrightarrow{\text{$R_1 := R_1 + (-2) \cdot R_3 $}}
\left(
\begin{array}{ccc|c}
1 & -3 & -6 & 0   \\
0 & 1 & 1 & 0 \\
0 & 0 & 0 & 1   \\
\end{array}
\right)
\xrightarrow{\text{$R_1 := R_1 + 3 \cdot R_2 $}}
\left(
\begin{array}{ccc|c}
1 & 0 & -3 & 0   \\
0 & 1 & 1 & 0 \\
0 & 0 & 0 & 1   \\
\end{array}
\right)
\]

The result of the process is the following reduced echelon matrix:

\[
\left(
\begin{array}{ccc|c}
1 & 0 & -3 & 0   \\
0 & 1 & 1 & 0 \\
0 & 0 & 0 & 1   \\
\end{array}
\right)
\]
The reduced echelon matrix shows us that there is no solution, because equation 3 will be as follows:
\begin{align*}
  0 \cdot x_1 + 0 \cdot x_2 + 0 \cdot x_3 &= 1 \\
  0 + 0 + 0 &= 1  \\
  0 &= 1
\end{align*}

Since $0 \neq 1$ the equation has no solutions. This means the equations are inconsistent.

\section*{Exercise 3}
\begin{align*}
2x + y + 2v + w &= 1 \\
4x + 4y + 6v + w &= 2 \\
6x + y + 4v + 5w &= 4 \\
2x + 3y + 5v + w &= 4 \\
\end{align*}


When we put the values of the equations into a coefficient matrix, it would look like this: \\
\[
\left(
\begin{array}{cccc}
2 & 1 & 2 & 1   \\
4 & 4 & 6 & 1 \\
6 & 1 & 4 & 5    \\
2 & 3 & 5 & 1
\end{array}
\right)
\]

The augmented matrix would be:
\[
\left(
\begin{array}{cccc|c}
2 & 1 & 2 & 1 & 1  \\
4 & 4 & 6 & 1 & 2  \\
6 & 1 & 4 & 5 & 4  \\
2 & 3 & 5 & 1 & 4
\end{array}
\right)
\]

In order to solve the equations we should try to get the matrix to (reduced) echelon form.
\[
\left(
\begin{array}{cccc|c}
2 & 1 & 2 & 1 & 1  \\
4 & 4 & 6 & 1 & 2  \\
6 & 1 & 4 & 5 & 4  \\
2 & 3 & 5 & 1 & 4
\end{array}
\right)
\xrightarrow{\text{$R_1 := \frac{1}{2} \cdot R_1$}}
\left(
\begin{array}{cccc|c}
1 & \frac{1}{2} & 1 & \frac{1}{2} & \frac{1}{2}  \\
4 & 4 & 6 & 1 & 2  \\
6 & 1 & 4 & 5 & 4  \\
2 & 3 & 5 & 1 & 4
\end{array}
\right)
\xrightarrow{\text{$R_2 := R_2 - 4 \cdot R_1$}}
\left(
\begin{array}{cccc|c}
1 & \frac{1}{2} & 1 & \frac{1}{2} & \frac{1}{2}  \\
0 & 2 & 2 & -1 & 0  \\
6 & 1 & 4 & 5 & 4  \\
2 & 3 & 5 & 1 & 4
\end{array}
\right)
\]
\[
\xrightarrow{\text{$R_3 := R_3 - 6 \cdot R_1$}}
\left(
\begin{array}{cccc|c}
1 & \frac{1}{2} & 1 & \frac{1}{2} & \frac{1}{2}  \\
0 & 2 & 2 & -1 & 0  \\
0 & -2 & -2 & 2 & 1  \\
2 & 3 & 5 & 1 & 4
\end{array}
\right)
\xrightarrow{\text{$R_4 := R_4 - 2 \cdot R_1$}}
\left(
\begin{array}{cccc|c}
1 & \frac{1}{2} & 1 & \frac{1}{2} & \frac{1}{2}  \\
0 & 2 & 2 & -1 & 0  \\
0 & -2 & -2 & 2 & 1  \\
0 & 2 & 3 & 0 & 3
\end{array}
\right)
\]
\[
\xrightarrow{\text{$R_3 := R_3 + 1 \cdot R_4$}}
\left(
\begin{array}{cccc|c}
1 & \frac{1}{2} & 1 & \frac{1}{2} & \frac{1}{2}  \\
0 & 2 & 2 & -1 & 0  \\
0 & 0 & 1 & 2 & 4  \\
0 & 2 & 3 & 0 & 3
\end{array}
\right)
\xrightarrow{\text{$R_3 := R_3 + 1 \cdot R_4$}}
\left(
\begin{array}{cccc|c}
1 & \frac{1}{2} & 1 & \frac{1}{2} & \frac{1}{2}  \\
0 & 2 & 2 & -1 & 0  \\
0 & 0 & 1 & 2 & 4  \\
0 & 2 & 3 & 0 & 3
\end{array}
\right)
\]
\[
\xrightarrow{\text{$R_4 := R_4 - 1 \cdot R_2$}}
\left(
\begin{array}{cccc|c}
1 & \frac{1}{2} & 1 & \frac{1}{2} & \frac{1}{2}  \\
0 & 2 & 2 & -1 & 0  \\
0 & 0 & 1 & 2 & 4  \\
0 & 0 & 1 & 1 & 3
\end{array}
\right)
\xrightarrow{\text{$R_4 := R_4 - 1 \cdot R_3$}}
\left(
\begin{array}{cccc|c}
1 & \frac{1}{2} & 1 & \frac{1}{2} & \frac{1}{2}  \\
0 & 2 & 2 & -1 & 0  \\
0 & 0 & 1 & 2 & 4  \\
0 & 0 & 0 & -1 & -1
\end{array}
\right)
\]
\[
\xrightarrow{\text{$R_4 := (-1) \cdot R_4$}}
\left(
\begin{array}{cccc|c}
1 & \frac{1}{2} & 1 & \frac{1}{2} & \frac{1}{2}  \\
0 & 2 & 2 & -1 & 0  \\
0 & 0 & 1 & 2 & 4  \\
0 & 0 & 0 & 1 & 1
\end{array}
\right)
\xrightarrow{\text{$R_3 := R_3 - 2 \cdot R_4$}}
\left(
\begin{array}{cccc|c}
1 & \frac{1}{2} & 1 & \frac{1}{2} & \frac{1}{2}  \\
0 & 2 & 2 & -1 & 0  \\
0 & 0 & 1 & 0 & 2  \\
0 & 0 & 0 & 1 & 1
\end{array}
\right)
\]
\[
\xrightarrow{\text{$R_2 := R_2 + 1 \cdot R_4$}}
\left(
\begin{array}{cccc|c}
1 & \frac{1}{2} & 1 & \frac{1}{2} & \frac{1}{2}  \\
0 & 2 & 2 & 0 & 1  \\
0 & 0 & 1 & 0 & 2  \\
0 & 0 & 0 & 1 & 1
\end{array}
\right)
\xrightarrow{\text{$R_2 := \frac{1}{2} R_2$}}
\left(
\begin{array}{cccc|c}
1 & \frac{1}{2} & 1 & \frac{1}{2} & \frac{1}{2}  \\
0 & 1 & 1 & 0 & \frac{1}{2}  \\
0 & 0 & 1 & 0 & 2  \\
0 & 0 & 0 & 1 & 1
\end{array}
\right)
\]
\[
\xrightarrow{\text{$R_2 := R_2 - 1 \cdot R_3$}}
\left(
\begin{array}{cccc|c}
1 & \frac{1}{2} & 1 & \frac{1}{2} & \frac{1}{2}  \\
0 & 1 & 0 & 0 & - \frac{3}{2}  \\
0 & 0 & 1 & 0 & 2  \\
0 & 0 & 0 & 1 & 1
\end{array}
\right)
\xrightarrow{\text{$R_1 := R_1 - \frac{1}{2} \cdot R_4$}}
\left(
\begin{array}{cccc|c}
1 & \frac{1}{2} & 1 & 0 & 0  \\
0 & 1 & 0 & 0 & - \frac{3}{2}  \\
0 & 0 & 1 & 0 & 2  \\
0 & 0 & 0 & 1 & 1
\end{array}
\right)
\]
\[
\xrightarrow{\text{$R_1 := R_1 - 1 \cdot R_3$}}
\left(
\begin{array}{cccc|c}
1 & \frac{1}{2} & 0 & 0 & -2  \\
0 & 1 & 0 & 0 & - \frac{3}{2}  \\
0 & 0 & 1 & 0 & 2  \\
0 & 0 & 0 & 1 & 1
\end{array}
\right)
\xrightarrow{\text{$R_1 := R_1 - \frac{1}{2} \cdot R_2$}}
\left(
\begin{array}{cccc|c}
1 & 0 & 0 & 0 & - \frac{5}{4}  \\
0 & 1 & 0 & 0 & - \frac{3}{2}  \\
0 & 0 & 1 & 0 & 2  \\
0 & 0 & 0 & 1 & 1
\end{array}
\right)
\]

After transforming the matrix we have the following reduced echelon matrix.
\[
\left(
\begin{array}{cccc|c}
1 & 0 & 0 & 0 & - \frac{5}{4}  \\
0 & 1 & 0 & 0 & - \frac{3}{2}  \\
0 & 0 & 1 & 0 & 2  \\
0 & 0 & 0 & 1 & 1
\end{array}
\right)
\]

The reduced echelon matrix shows us that there is a unique solution, so the equations are consistent. When we fill in the results of the reduced echelon matrix in the original equations, we see that they hold.\\
We see that the first equation results the correct result of 1.
\begin{align*}
2 \cdot - \frac{5}{4} + 1 \cdot - \frac{3}{2} + 2 \cdot 2 + 1 \cdot 1
&= - \frac{10}{4} - \frac{3}{2} + 4 + 1 \\
&= - \frac{10}{4} - \frac{6}{4} + \frac{16}{4} + \frac{4}{4} \\
&= - \frac{16}{4} + \frac{16}{4} + \frac{4}{4} \\
&= \frac{4}{4} \\
&= 1
\end{align*}

When we fill in the second equation we see this doesn't hold.
\begin{align*}
4 \cdot - \frac{5}{4} + 4 \cdot - \frac{3}{2} + 6 \cdot 2 + 1 \cdot 1
&= - \frac{20}{4} - \frac{12}{2} + 12 + 1 \\
&= - \frac{20}{4} - \frac{24}{4} + 12 + 1 \\
&= - 5 - 6 + 12 + 1 \\
&= 13 -11 \\
&= 2
\end{align*}

The third equation also results in the correct result of 4.
\begin{align*}
6 \cdot - \frac{5}{4} + 1 \cdot - \frac{3}{2} + 4 \cdot 2 + 5 \cdot 1
&= - \frac{30}{4} - \frac{3}{2} + 8 + 5 \\
&= - \frac{15}{2} - \frac{3}{2} + \frac{16}{2} + \frac{10}{2} \\
&= - \frac{18}{2} + \frac{26}{2} \\
&= 13 - 9
&= 4
\end{align*}

Finally, we see that the fourth equation gives the correct result of 4:
\begin{align*}
2 \cdot - \frac{5}{4} + 3 \cdot - \frac{3}{2} + 5 \cdot 2 + 1 \cdot 1
&= - \frac{10}{4} - \frac{9}{2} + 10 + 1 \\
&= - \frac{5}{2} - \frac{9}{2} + 10 + 1 \\
&= - \frac{17}{2} + 11 \\
&= 11 - 7 \\
&= 4
\end{align*}

\section*{Exercise 4}
\begin{align*}
  3x_1 + x_2 + x_3 &= 3 \\
  x_1 - x_2 -x_3 &= 1
\end{align*}

% In order to find the vectors and an intersecting point we should first solve the equations.
% \[
% \left(
% \begin{array}{ccc|c}
% 3 & 1 & 1 & 3   \\
% 1 & -1 & -1 & 1 \\
% \end{array}
% \right)
% \]
%
% \[
% \left(
% \begin{array}{ccc|c}
% 3 & 1 & 1 & 3   \\
% 1 & -1 & -1 & 1 \\
% \end{array}
% \right)
% \xrightarrow{\text{$R_2 := R_2 - 3 \cdot R_1$}}
% \left(
% \begin{array}{ccc|c}
% 3 & 1 & 1 & 3   \\
% 0 & -4 & -4 & -8 \\
% \end{array}
% \right)
% \]
% \[
% \xrightarrow{\text{$R_2 := (- \frac{1}{4})R_2$}}
% \left(
% \begin{array}{ccc|c}
% 3 & 1 & 1 & 3   \\
% 0 & 1 & 1 & 2 \\
% \end{array}
% \right)
% \xrightarrow{\text{$R_1 := R_1 - 1 \cdot R_2$}}
% \left(
% \begin{array}{ccc|c}
% 3 & 0 & 0 & 1   \\
% 0 & 1 & 1 & 2 \\
% \end{array}
% \right)
% \]
% \[
% A_{m,n} =
%  \begin{pmatrix}
%   a_{1,1} & a_{1,2} & \cdots & a_{1,n} \\
%   a_{2,1} & a_{2,2} & \cdots & a_{2,n} \\
%   \vdots  & \vdots  & \ddots & \vdots  \\
%   a_{m,1} & a_{m,2} & \cdots & a_{m,n}
%  \end{pmatrix}
%  \]
%
%
%  \[
%  \left(
%  \begin{array}{cccc|c}
%  1 & 0 & 3 & -1 & 0 \\
%  0 & 1 & 1 & -1 & 0 \\
%  0 & 0 & 0 & 0 & 0 \\
%  \end{array}
%  \right)
%  \]


\end{document}
