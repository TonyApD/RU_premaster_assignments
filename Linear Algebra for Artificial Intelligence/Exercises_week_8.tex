%
% Although we try to provide a template that completely
% matches the corresponding assignment, we do expect you
% to check that you have indeed answered all questions.
%

% ALSO VERY IMPORTANT:
% This is just a template to help you with the LaTeX part of the assignment.
% So you may change it completely according to your own wishes!
%

\documentclass[a4paper]{article}
% Typically the 'article' class is appropriate for assignments.
% And we print it on a4, so we include that as well.

\usepackage{a4wide}
% To decrease the margins and allow more text on a page.

\usepackage{graphicx}
% To deal with including pictures.

\usepackage{enumerate}
% To provide a little bit more functionality than with LaTeX's default
% enumerate environment.

\usepackage{array}
% To provide a little bit more functionality than with LaTeX's default
% array environment.

\usepackage[american]{babel}
% Use this if you want to write the document in US English. It takes care of
% (usually) proper hyphenation.
% If you want to write your answers in Dutch, please replace 'american'
% by 'dutch'.
% Note that after a change it may be that the first compilation of LaTeX
% fails. That is normal and caused by the fact that in auxiliary files
% from previous runs, there may still be a \selectlanguage{american}
% around, which is invalid if 'american' is not incorporated with babel.

\usepackage{amssymb}
% This package loads mathematical things like the fonts for the blackboard
% bold for the set of natural numbers.
\usepackage{amsmath}
% And some student asked me to include amsmath as well...

\usepackage{tikz}
\usetikzlibrary{arrows}
\usetikzlibrary{positioning}
% The tikz package can be used to draw all kinds of diagrams.
% In this assignment it is being used for drawing the parse trees


\usepackage{xspace}
% xspace can be used to let LaTeX decide whether a command should be followed
% y a space or not, depending on what follows.

% Some obscure definition to create a circled node within xy.
% The definition is made by Freek Wiedijk who prefers to do his definitions
% in TeX instead of LaTeX, which explains the \def instead of \newcommand.
\def\node{*++[o][F-]}
\def\fnode{*++[o][F=]}

\usepackage{csquotes}
\usepackage{mathtools}
\usepackage{cancel}

% This command puts a `def' on top of an `='.
\newcommand{\isdef}{\ensuremath{\,\,\buildrel\rm def\over=}\,\,}

\reversemarginpar
\title{Linear Algebra for AI\\Assignment 8}

\author{Tony Lopar \\ s1013792 \\ Group 5 \quad Felicity Reddel}

\begin{document}
\maketitle

% \left(
% \begin{array}{ccc}
% 8 & 2 & \frac{5}{2} \\
% 1 & 6 & 5  \\
% 1 & 2 & \frac{5}{2} \\
% \end{array}
% \right)

\section*{Exercise 1}
\[
v =
\left(
\begin{array}{c}
2 \\
3 \\
-6 \\
\end{array}
\right)
\enspace
w =
\left(
\begin{array}{c}
1 \\
-5 \\
2 \\
\end{array}
\right)
\]
\begin{enumerate}
  \item We have the vectors $v = (2, 3, -6)$ and $w = (1, -5, 2)$. The projection of v onto w can be computed by finding a multiple pw of w for which the distance d(pw, v) is minimal. This can be done using the formula for projection on v.
  % The distance between pw and v is:
  % \[
  % d(pw, v) = ||pw - v|| = \sqrt{(1p - 2)^2 + (-5p - 3)^2 + (2p + 6)^2}
  % \]
  % So, we need to find a p for which $(1p - 2)^2 + (-5p - 3)^2 + (2p + 6)^2$ is minimal. We can start calculating:
  % \begin{align*}
  %   (1p - 2)^2 + (-5p - 3)^2 + (2p + 6)^2 &= p^2 - 4p + 4 + 25p^2 + 30p + 9 + 4p^2 + 24p + 36 \\
  %   &= 30p^2 + 50p + 49
  % \end{align*}
  % In order to find the minimum value of p we should derive the equation and check where the derivation is equal to zero.
  \[
  v_{||} = pw = \frac{\langle v, w \rangle}{\langle w, w \rangle}w = \frac{2 - 15 - 12}{1 + 25 + 4}(1, -5, 2) = \frac{-25}{30}(1, -5, 2) = - \frac{5}{6}(1, -5, 2)
  \]
  \item We should find a vector which is orthogonal for both v and w. We may compute this by taking the cross product of the two vectors.
  \[
  v \times w = (v_2w_3 - v_3w_2) - (v_1w_3 - v_3w_1) + (v_1w_2 - v_2w_1) = (6 - 30) - (4 + 6) + (-10 - 3) = (-24, -10, -13)
  \]
  So, a possible vector for u is (-24, -10, -13). We can check whether this vector is orthogonal to the two other vectors. First, we check whether is orthogonal with v.
  \[
  \langle u, v \rangle = -24 \cdot 2 +  -10 \cdot 3 + -13 \cdot - 6 = -48 - 30 + 78 = 0
  \]
  So, v and u are orthogonal. Now, we should check whether u and w are also orthogonal.
  \[
  \langle u, w \rangle = -24 \cdot 1 +  -10 \cdot -5 + -13 \cdot 2 = -24 + 50 - 26 = 0
  \]
  So, now we see that u is orthogonal to both v and w.
\end{enumerate}

\section*{Exercise 2}
\[
v_1 =
\left(
\begin{array}{c}
1 \\
1 \\
1 \\
0 \\
\end{array}
\right)
v_2 =
\left(
\begin{array}{c}
0 \\
1 \\
1 \\
1 \\
\end{array}
\right)
v_3 =
\left(
\begin{array}{c}
1 \\
1 \\
1 \\
1 \\
\end{array}
\right)
\]
\begin{enumerate}
  \item We can do this with Gram-schmidt orthogonalisation. Where we take the independent set $\{v_1, v_2, v_3\}$ as starting point.
  \begin{align*}
    v_1' &= v_1 = (1, 1, 1, 0) \\
    v_2' &= v_2 - \frac{\langle v_2, v_1' \rangle}{\langle v_1', v_1' \rangle}v_1' = (0, 1, 1, 1) -  \frac{2}{3}(1, 1, 1, 0) = (-\frac{2}{3}, \frac{1}{3}, \frac{1}{3}, 1) \\
    v_3' &= v_3 - \frac{\langle v_3, v_1' \rangle}{\langle v_1', v_1' \rangle}v_1' - \frac{\langle v_3, v_2' \rangle}{\langle v_2', v_2' \rangle}v_2' = (1, 1, 1, 1) - \frac{3}{3}(1, 1, 1, 0) - \frac{1}{3}(-\frac{2}{3}, \frac{1}{3}, \frac{1}{3}, 1) \\ &= (0, 0, 0, 1) - \frac{1}{3}(-\frac{2}{3}, \frac{1}{3}, \frac{1}{3}, 1) = (0, 0, 0, 1) - (-\frac{2}{9}, \frac{1}{9}, \frac{1}{9}, \frac{3}{9}) = (\frac{2}{9}, -\frac{1}{9}, -\frac{1}{9}, \frac{6}{9})
  \end{align*}
  \item We may eliminate the fractions from $v_2'$ by multiplying the values with 3. This gives $v_2' = (-2, 1, 1, 3)$. The fractions from $v_3'$ may also be removed by multiplying with 9 which gives $v_3' = (2, -1, -1, 6)$. Now we can check whether the vectors are indeed orthogonal.
  \begin{align*}
    \langle v_1', v_2' \rangle &= 1 \cdot - 2 + 1 \cdot 1 + 1 \cdot 1 + 0 \cdot 3 = -2 + 1 + 1 + 0 = 0 \\
    \langle v_1', v_3' \rangle &= 1 \cdot 2 + 1 \cdot -1 + 1 \cdot -1 + 0 \cdot 6 = 2 - 1 - 1 + 0 = 0 \\
    \langle v_2', v_3' \rangle &= -2 \cdot 2 + 1 \cdot -1 + 1 \cdot -1 + 3 \cdot 6 = -4 - 1 - 1 + 18 = 12
  \end{align*}
  I think I made a miscalculation somewhere, since $v_2'$ and $v_3'$ are not orthogonal.
  \item
  \item The distances between u and the three given vectors are as follows:
  \begin{align*}
  d(u, v_1) &= ||u - v_1|| = \sqrt{(2 - 1)^2 + (2 - 1)^2 + (3 - 1)^2 + (1 - 0)^2} = \sqrt{1 + 1 + 4 + 1} = \sqrt{7} \\
  d(u, v_2) &= ||u - v_2|| = \sqrt{(2 - 0)^2 + (2 - 1)^2 + (3 - 1)^2 + (1 - 1)^2} = \sqrt{4 + 1 + 4 + 0} = \sqrt{9} \\
  d(u, v_3) &= ||u - v_3|| = \sqrt{(2 - 1)^2 + (2 - 1)^2 + (3 - 1)^2 + (1 - 1)^2} = \sqrt{1 + 1 + 4 + 0} = \sqrt{6}
  \end{align*}
  If we compare this with $||u - u_0||$ I think the distance with $u_0$ should be smaller than the distance with any other vector. The reason for this is that the pojection is a vector with a minimal distance. No other vectors may have a shorter distance to u.

\end{enumerate}

\section*{Exercise 3}
\begin{enumerate}
  \item
\end{enumerate}

\end{document}
